\apendice{Documentación técnica de programación}

\section{Introducción}

En este apéndice se presenta la documentación técnica del proyecto, proporcionando una descripción detallada de la estructura del código y sus directorios. Este apartado tiene como objetivo facilitar la comprensión del proyecto para desarrolladores que deseen trabajar en él, analizarlo o realizar modificaciones futuras.

\section{Estructura de directorios}

Todo el código de este proyecto se encuentra dentro de la carpeta \texttt{code}, que se divide en dos subcarpetas principales:

\begin{itemize}
    \item \texttt{app/} -- Contiene toda la aplicación web desarrollada para el proyecto, organizada siguiendo el patrón Modelo-Vista-Controlador (MVC). Incluye el código relacionado con los controladores, modelos, vistas y utilidades, así como los recursos estáticos necesarios para la interfaz de usuario.
    \item \texttt{Research and experiments/} -- Agrupa todas las pruebas y experimentos realizados durante el desarrollo. En esta carpeta se incluyen tanto los prototipos iniciales como los análisis realizados con diferentes componentes del proyecto. Solo se han conservado aquellas pruebas que se consideran claras y relevantes.
\end{itemize}

Además, la documentación generada durante el desarrollo se encuentra en la carpeta \texttt{docs/latex}.

A continuación, se describe la estructura interna en detalle, destacando las funcionalidades principales de cada archivo o directorio:

\subsection*{Estructura de la aplicación}
            
\begin{itemize}
    \item \texttt{code/}
    \begin{itemize}
        \item \texttt{app/} -- Directorio principal de la aplicación web.
        \item \texttt{Research and experiments/} -- Directorio que contiene experimentos y pruebas realizadas durante el desarrollo.
    \end{itemize}
\end{itemize}

\begin{itemize}
	\item \texttt{app/}
	\begin{itemize}
		\item \texttt{controllers/} -- Contiene la lógica principal de control que conecta los modelos con las vistas.
        	\begin{itemize}
        		\item \texttt{\_\_init\_\_.py} -- Archivo de inicialización del módulo.
        		\item \texttt{callbacks.py} -- Define los callbacks utilizados en la aplicación, gestionando la interacción entre la interfaz de usuario y la lógica del servidor.
        		\item \texttt{clustering.py} -- Implementa funciones relacionadas con la ejecución de algoritmos de agrupamiento, incluyendo TRACLUS.
 		\end{itemize}
        	\item \texttt{models/} -- Agrupa la lógica de procesamiento de datos y las funciones principales del algoritmo TRACLUS.
         \begin{itemize}
         	\item \texttt{\_\_init\_\_.py} -- Archivo de inicialización del módulo.
         	\item \texttt{data\_processing.py} -- Contiene funciones para la limpieza, transformación y preparación de los datos.
         	\item \texttt{mapping.py} -- Proporciona herramientas para la visualización de datos geoespaciales.
         	\item \texttt{TRACLUS.py} -- Implementa el núcleo del algoritmo TRACLUS.
		\end{itemize}
        	\item \texttt{saved\_results/} -- Almacena resultados generados y guardados por el usuario.
		\item \texttt{source/} -- Contiene los archivos fuente necesarios para generar la documentación con Sphinx.
		\begin{itemize}
    			\item \texttt{Makefile} -- Archivo de configuración para automatizar la generación de la documentación. Se usa con el comando \texttt{make}.
   			\item \texttt{conf.py} -- Archivo de configuración principal para Sphinx. Define las opciones y extensiones utilizadas para generar la documentación.
    			\item \texttt{index.rst} -- Página principal de la documentación, que actúa como índice para los demás archivos documentados.
   			\item \texttt{modules.rst} -- Archivo que agrupa la documentación de los módulos del proyecto, organizados según la estructura de las carpetas.
    			\item \texttt{some\_other\_doc.rst} -- Archivo adicional que puede contener documentación específica o guías de usuario.
    			\item \texttt{\_static/} -- Carpeta reservada para archivos estáticos como imágenes, hojas de estilo personalizadas o archivos JS. Es importante para mantener el diseño y la estructura de la documentación HTML.
		\end{itemize}
        	\item \texttt{test/} -- Contiene pruebas unitarias y funcionales.
        	\begin{itemize}
        		\item \texttt{\_\_init\_\_.py} -- Archivo de inicialización del módulo de pruebas.
        		\item \texttt{test\_mapping.py} -- Pruebas específicas para la representación de datos en los mapa.
        		\item \texttt{test\_TRACLUS.py} -- Pruebas específicas para la ejecución del algoritmo TRACLUS.
		\end{itemize}
        	\item \texttt{utils/} -- Incluye utilidades generales y configuraciones globales.
        	\begin{itemize}
        		\item \texttt{\_\_init\_\_.py} -- Archivo de inicialización del módulo.
         	\item \texttt{config.py} -- Configuraciones generales del proyecto, como rutas y parámetros predeterminados.
         	\item \texttt{data\_saveload.py} -- Funciones para guardar y cargar los datos de los experimentos ya ejecutados.
        		\item \texttt{data\_utils.py} -- Funciones auxiliares para manipulación de datos.
        	\end{itemize}
        	\item \texttt{views/} -- Directorio que agrupa las vistas y la estructura visual de la aplicación.
 		\begin{itemize}
        		\item \texttt{assets/} -- Contiene recursos estáticos como hojas de estilo y archivos multimedia.
			\begin{itemize}
            		\item \texttt{style.css} -- Archivo CSS para la personalización del diseño de la interfaz de usuario.
            	\end{itemize}
			\item \texttt{layout/} -- Define las vistas de cada sección de la aplicación.
			\begin{itemize}
            		\item \texttt{\_\_init\_\_.py} -- Archivo de inicialización del módulo.
           		\item \texttt{datauptate\_page.py} -- Página para la carga y actualización de datos.
            		\item \texttt{experiment\_page.py} -- Página para seleccionar algoritmos de clustering.
            		\item \texttt{map\_page.py} -- Página que muestra mapas con los datos sin tratar.
            		\item \texttt{navbar.py} -- Barra de navegación, permite cambiar entre páginas y descargar los datos que estén siendo visualizados.
            		\item \texttt{select\_page.py} -- Página para la selección del experimento o la creación de uno nuevo.
            		\item \texttt{table\_page.py} -- Página para mostrar tablas con los resultados del análisis.
				\item \texttt{TRACLUSmap\_page.py} -- Página que muestra mapas con los resultados del algoritmo TRACLUS.
			\end{itemize}
		\end{itemize}
		\item \texttt{\_\_init\_\_.py} -- Archivo de inicialización del módulo principal.
		\item \texttt{main.py} -- Archivo principal que ejecuta la aplicación web.
		\item \texttt{requirements.txt} -- Archivo que especifica las dependencias necesarias para ejecutar el proyecto.
    \end{itemize}
\end{itemize}

\subsection*{Estructura de las investigaciones}

\begin{itemize}
    \item \texttt{Research and experiments/}
    \begin{itemize}
        \item \texttt{Clusters Algorit test/}
        \begin{itemize}
            \item \texttt{Clusters\_algorit\_test.ipynb} -- Funciones que permitieron comparar los diferentes algoritmos antes de su implementación en la página web.
        \end{itemize}
        \item \texttt{Data comparative test/}
        \begin{itemize}
            \item \texttt{Data\_comparative\_test.ipynb} -- Funciones desarrolladas para representar los datos y comparar los resultados obtenidos.
        \end{itemize}
        \item \texttt{Data transformer/}
        \begin{itemize}
            \item \texttt{Geolife\_analizer.ipynb} -- Funciones que transforman los datos del conjunto de datos Geolife \cite{geolife_trajectories} al formato correcto para utilizar con el algoritmo TRACLUS.
            \item \texttt{Movebank\_analizer.ipynb} -- Funciones que transforman los datos del conjunto de datos Movebank \cite{movebank} al formato correcto para utilizar con el algoritmo TRACLUS.
        \end{itemize}
        \item \texttt{TRACLUS test/}
        \begin{itemize}
            \item \texttt{Geolife\_test.ipynb} -- Pruebas para la creación de funciones de lectura y representación de datos.
            \item \texttt{Optimization\_TRACLUS.ipynb} -- Pruebas realizadas para la optimización del algoritmo TRACLUS.
        \end{itemize}
    \end{itemize}
\end{itemize}


\subsection*{Estructura de la documentación}

\begin{itemize}
	\item \texttt{docs/}
	\begin{itemize}
		\item \texttt{latex/}
		\begin{itemize}
   			\item \texttt{img/} -- Imágenes utilizadas en la documentación.
    			\item \texttt{tex/} -- Archivos \LaTeX{} para generar documentos técnicos.
    			\item \texttt{anexos.pdf} -- Documento de anexos generado.
    			\item \texttt{anexos.tex} -- Código fuente del documento de anexos.
    			\item \texttt{bibliografia.bib} -- Archivo BibTeX con referencias bibliográficas principales.
    			\item \texttt{bibliografiaAnexos.bib} -- Archivo BibTeX con referencias adicionales.
    			\item \texttt{memoria.pdf} -- Documento principal del proyecto.
    			\item \texttt{memoria.tex} -- Código fuente del documento principal.
		\end{itemize}
	\end{itemize}
\end{itemize}


\section{Manual del programador}

Este apartado proporciona información sobre las herramientas clave que un programador necesita para trabajar en el proyecto de manera eficiente, desde la instalación hasta la configuración del entorno de desarrollo.

\subsection{Herramientas clave}

A continuación, se detallan las herramientas y su configuración:

\subsubsection{Git}
Git es esencial para gestionar el control de versiones del proyecto.

\begin{enumerate}
    \item Descargue e instale Git desde la página oficial: \url{https://git-scm.com/}.
    \item Configure su nombre de usuario y correo electrónico:
    \begin{verbatim}
    git config --global user.name "TuNombre"
    git config --global user.email "TuCorreo@example.com"
    \end{verbatim}
    \item Clone el repositorio del proyecto:
    \begin{verbatim}
    git clone https://github.com/agd1017/TFG_TRACLUS.git
    \end{verbatim}
\end{enumerate}

\subsubsection{Visual Studio Code}
Visual Studio Code es el editor recomendado.

\begin{enumerate}
    \item Descargue VS Code desde: \url{https://code.visualstudio.com/}.
    \item Instale las extensiones sugeridas:
    \begin{itemize}
        \item Python.
        \item Pylance.
        \item GitLens.
    \end{itemize}
    \item Configure el intérprete de Python para que apunte a su entorno virtual.
\end{enumerate}

\subsubsection{Python}
El proyecto requiere Python 3.11.2 o superior.

\begin{enumerate}
    \item Descargue Python desde: \url{https://www.python.org/}.
    \item Durante la instalación, habilite la opción \textit{Add Python to PATH}.
    \item Verifique la instalación con:
    \begin{verbatim}
    python --version
    \end{verbatim}
\end{enumerate}

\subsubsection{Librerías Python}
El archivo \texttt{requirements.txt} especifica las dependencias necesarias.

\begin{enumerate}
    \item Cree un entorno virtual:
    \begin{verbatim}
    python -m venv venv
    \end{verbatim}
    \item Active el entorno virtual:
    \begin{itemize}
        \item En Windows:
        \begin{verbatim}
        venv\Scripts\activate
        \end{verbatim}
        \item En macOS/Linux:
        \begin{verbatim}
        source venv/bin/activate
        \end{verbatim}
    \end{itemize}
    \item Instale las dependencias:
    \begin{verbatim}
    pip install -r requirements.txt
    \end{verbatim}
\end{enumerate}

\subsubsection{Documentación sphinx html}

Para generar y ver la documentación de este proyecto utilizando Sphinx, sigue los siguientes pasos:

\begin{enumerate}
    \item \textbf{Instalar las dependencias:}
    Asegúrate de que tienes Sphinx instalado en tu entorno de desarrollo. Si no lo tienes, puedes instalarlo utilizando \texttt{pip}:

    \begin{verbatim}
    pip install sphinx
    \end{verbatim}

    Además, asegúrate de que todas las dependencias del proyecto estén instaladas, lo que puede incluir otras librerías necesarias para la documentación.

    \item \textbf{Navegar al directorio de la documentación:}
    Ve al directorio donde se encuentra la carpeta \texttt{source}, que contiene la documentación generada por Sphinx. Normalmente, se encuentra en el directorio raíz del proyecto.

    \begin{verbatim}
    cd /ruta/a/tu/proyecto/source
    \end{verbatim}

    \item \textbf{Generar la documentación HTML:}
    Dentro del directorio \texttt{source}, utiliza el archivo \texttt{Makefile} para generar la documentación en formato HTML. Puedes hacerlo ejecutando el siguiente comando en tu terminal:

    \begin{verbatim}
    make html
    \end{verbatim}
    
    O el archivo \texttt{make.bat} si \texttt{Makefile} no funciona:
    
    \begin{verbatim}
    ./make.bat html
    \end{verbatim}

    Este comando generará los archivos HTML en el directorio \texttt{\_build/html}.

    \item \textbf{Ver la documentación:}
    Después de ejecutar \texttt{make html}, abre el archivo \texttt{index.html} que se encuentra en la carpeta \texttt{\_build/html} con tu navegador web favorito para ver la documentación generada.

    \begin{verbatim}
    start _build/html/index.html
    \end{verbatim}

    \item \textbf{Personalización:}
    Si deseas modificar la apariencia o el contenido de la documentación, puedes editar los archivos en la carpeta \texttt{source}, como \texttt{index.rst}, \texttt{modules.rst}, y otros archivos de documentación. Tras realizar cambios, simplemente vuelve a ejecutar el comando \texttt{make html} para ver las actualizaciones.
\end{enumerate}


\section{Compilación, instalación y ejecución del proyecto}

En este apartado se detallan los pasos para ejecutar el proyecto en un entorno local desde el código fuente.

\subsection{Preparación del entorno}
\begin{enumerate}
    \item Asegúrese de haber instalado todas las herramientas clave mencionadas en el \textit{Manual del programador}.
    \item Clone el repositorio del proyecto en su máquina local.
    \item Navegue hasta la carpeta \texttt{code/app}.
    \item Cree y active el entorno virtual como se describe en la sección de \texttt{Librerías Python}.
\end{enumerate}

\subsection{Ejecución del proyecto}
\begin{enumerate}
    \item Ejecute el archivo principal de la aplicación:
    \begin{verbatim}
    python main.py
    \end{verbatim}
    \item Abra un navegador web y acceda a:
    \begin{verbatim}
    http://127.0.0.1:8050/
    \end{verbatim}
\end{enumerate}

\subsection{Modificaciones y control de versiones}
Para realizar cambios en el código:
\begin{enumerate}
    \item Edite los archivos necesarios utilizando Visual Studio Code.
    \item Guarde los cambios y verifique que la aplicación funcione correctamente.
    \item Utilice Git para gestionar los cambios:
    \begin{verbatim}
    git add .
    git commit -m "Descripción de los cambios"
    git push origin rama-principal
    \end{verbatim}
\end{enumerate}

Con estas instrucciones, el programador podrá compilar, instalar y ejecutar el proyecto localmente de manera efectiva.

\section{Despliegue de la aplicación}

Para que la aplicación web funcionase correctamente y otros usuarios pudieran utilizarla, era necesario desplegarla desde el entorno local a un entorno de producción. Existen dos formas principales de realizar este proceso: crear un servidor propio o subir la aplicación a un servidor en la nube proporcionado por un tercero. Por razones económicas, se decidió optar por la segunda opción y utilizar un servidor en la nube. Para ello, se exploraron diversos servicios compatibles con aplicaciones basadas en Python.

Entre las opciones evaluadas se encontraron AWS, Azure, Heroku y Conduktor. Aunque muchas de estas plataformas ofrecían soluciones robustas con buen rendimiento, todas presentaban limitaciones económicas, ya que sus funcionalidades avanzadas suelen estar asociadas a costos recurrentes. Esto nos llevó a buscar servicios que ofrecieran una base gratuita que cubriese nuestras necesidades principales.

\subsection{Render: la solución elegida}

Tras evaluar diferentes opciones, se decidió utilizar \textbf{Render}, una plataforma que permite el despliegue de aplicaciones web, APIs y otros servicios. Render ofrece un plan gratuito que resulta ideal para proyectos pequeños o de desarrollo inicial, lo que lo convierte en una opción atractiva para aquellos con presupuestos limitados.

Render proporciona varias ventajas para aplicaciones como la nuestra:
\begin{itemize}
    \item \textbf{Compatibilidad con Python:} Es compatible con aplicaciones basadas en frameworks como Dash, Flask o Django.
    \item \textbf{Despliegue automático:} Permite integrar repositorios de GitHub o GitLab para desplegar automáticamente los cambios realizados en el código.
    \item \textbf{Certificados SSL gratuitos:} Render ofrece certificados de seguridad SSL para garantizar conexiones seguras.
    \item \textbf{Facilidad de configuración:} La plataforma cuenta con una interfaz intuitiva y bien documentada, lo que facilita el proceso de configuración incluso para usuarios con experiencia limitada en despliegues.
    \item \textbf{Soporte para aplicaciones persistentes:} Render soporta aplicaciones que requieren bases de datos o almacenamiento adicional, ideal para aplicaciones web interactivas.
\end{itemize}

\subsection{Proceso de despliegue en Render}

El despliegue de la aplicación en Render se realizó siguiendo los pasos descritos a continuación:

\begin{enumerate}
    \item \textbf{Preparación del repositorio:}
    \begin{itemize}
        \item El proyecto ya estaba alojado en un repositorio de GitHub, se incluyo un archivo \texttt{requirements.txt} que especifica las dependencias necesarias para ejecutar la aplicación.
    \end{itemize}

    \item \textbf{Creación del servicio en Render:}
    \begin{itemize}
        \item Se creó una cuenta gratuita en Render.
        \item Desde el panel de control, se seleccionó la opción \textit{"New Web Service"} y se vinculó el repositorio de GitHub al servicio.
    \end{itemize}

    \item \textbf{Configuración del entorno:}
    \begin{itemize}
        \item Se especificó el comando de inicio de la aplicación, como \texttt{python code/app/main.py}.
        \item Se configuraron las variables de entorno necesarias, como claves API o configuraciones específicas de la aplicación.
    \end{itemize}

    \item \textbf{Despliegue automático:}
    \begin{itemize}
        \item Render detectó automáticamente el contenido del repositorio y comenzó el proceso de construcción e implementación.
        \item Una vez finalizado el proceso, se asignó una URL pública para acceder a la aplicación.
    \end{itemize}

    \item \textbf{Pruebas en producción:}
    \begin{itemize}
        \item Se realizaron pruebas para verificar que todas las funcionalidades de la aplicación estuvieran operativas y que no existieran problemas relacionados con dependencias o configuración.
    \end{itemize}
\end{enumerate}

\subsection{Consideraciones finales}

Aunque Render fue una herramienta muy útil durante el desarrollo, especialmente por su facilidad de uso y la ausencia de necesidad de gestionar infraestructura propia, las limitaciones en capacidad de procesamiento y memoria debido al bajo presupuesto impidieron realizar pruebas significativas de la aplicación. Estas restricciones, derivadas del plan gratuito, afectaron el rendimiento al procesar grandes volúmenes de datos y al manejar operaciones intensivas.  

Para que la aplicación pueda ser probada y utilizada en todo su potencial, sería necesario aumentar considerablemente el presupuesto destinado a recursos de servidor, lo que permitiría un entorno de despliegue más robusto y acorde a los requisitos de la aplicación.

\section{Pruebas del sistema}

En esta sección se describirá cómo realizar pruebas en el sistema para comprobar que los cambios realizados se han implementado correctamente.

\subsection{Prueba manual}

Las pruebas más utilizadas en este proyecto, especialmente para verificar la ejecución del algoritmo y la funcionalidad general de la aplicación, son de tipo manual.

Para realizar las pruebas de manera correcta y en un entorno local, se recomienda modificar la función \texttt{app.run\_server()} ubicada al final del archivo \texttt{main.py}. 

Se deben realizar los siguientes ajustes:
\begin{itemize}
    \item Añadir el parámetro \texttt{debug=True}, lo que permitirá visualizar en la consola los errores o advertencias durante la ejecución.
    \item Configurar el servidor local especificando el host y el puerto. Por ejemplo:
    \begin{verbatim}
    app.run_server(debug=True, host='127.0.0.1', port=8050)
    \end{verbatim}
\end{itemize}

Después de realizar esta configuración, ejecute el archivo \texttt{main.py}. Se deberá acceder en un navegador web a la dirección \texttt{http://127.0.0.1:8050}. Si la aplicación se ha cargado correctamente, se mostrará la página inicial, desde la cual podrá navegar y utilizarla normalmente.

Para realizar cualquier prueba, siga el flujo de trabajo estándar de la aplicación para llegar al recurso que desea verificar. Por ejemplo:
\begin{itemize}
    \item Si el cambio realizado afecta a la visualización de los resultados tras cargar un experimento, intente acceder a uno de los experimentos ya creados. 
    \item Si el cambio afecta a la carga de datos, deberá crear nuevos experimentos, ya que los experimentos guardados anteriormente no reflejarán los cambios en la lógica de carga.
\end{itemize}

En caso de que no sea posible acceder a los experimentos guardados, asegúrese de que la configuración de rutas en el archivo \texttt{config.py} sea válida para su dispositivo. Si es necesario, actualice las rutas para que apunten correctamente a las ubicaciones de los datos y recursos requeridos.

Este método permite identificar errores directamente en el flujo de trabajo de la aplicación y verificar que los cambios realizados se comporten de manera esperada.

\subsection{Testing}

El proyecto también incluye pruebas automáticas que pueden ejecutarse para verificar la funcionalidad de diversos componentes. Estas pruebas están ubicadas en el directorio \texttt{test/}.

\subsubsection{Ejecución de pruebas}

Para ejecutar las pruebas existentes, siga estos pasos:
\begin{enumerate}
    \item Asegúrese de que todas las dependencias necesarias estén instaladas. Puede verificar esto ejecutando:
    \begin{verbatim}
    pip install -r requirements.txt
    \end{verbatim}
    \item Desde la terminal, navegue hasta el directorio raíz del proyecto donde se encuentra el archivo \texttt{main.py}.
    \item Ejecute las pruebas utilizando el siguiente comando:
    \begin{verbatim}
    pytest -v {nombre test}
    \end{verbatim}
    \item Si este no funcion use el siguiente: 
    \begin{verbatim}
    PYTHONPATH=code/app pytest -v code/app/test/{nombre test}
    \end{verbatim}
    \item Revise los resultados en la terminal. Si alguna prueba falla, se mostrará el motivo del fallo y la ubicación correspondiente en el código.
\end{enumerate}

\subsubsection{Creación de nuevas pruebas}

Si necesita añadir nuevas pruebas al sistema, siga estos pasos:
\begin{enumerate}
    \item Cree un nuevo archivo de prueba en el directorio \texttt{test/}. Por convención, los archivos de prueba deben comenzar con el prefijo \texttt{test\_}, por ejemplo: \texttt{test\_nueva\_funcionalidad.py}.
    \item Dentro del archivo, defina funciones que comiencen con el prefijo \texttt{test\_}, como en el siguiente ejemplo:
    \begin{verbatim}
    def test_nueva_funcionalidad():
        resultado = funcion_a_probar(parametros)
        assert resultado == valor_esperado
    \end{verbatim}
    \item Utilice las funciones de prueba para validar las salidas de las funciones o métodos que haya implementado. Asegúrese de cubrir casos de uso válidos, así como posibles errores o entradas no válidas.
    \item Ejecute nuevamente \texttt{pytest} para confirmar que todas las pruebas, incluidas las nuevas, se ejecutan correctamente.
\end{enumerate}

Al incluir nuevas pruebas, asegúrese de documentar el propósito de cada prueba dentro del archivo correspondiente. Esto ayudará a otros desarrolladores a comprender el objetivo y alcance de cada conjunto de pruebas.



