\apendice{Anexo de sostenibilización curricular}

\section{Introducción}
Este proyecto se alinea con varios Objetivos de Desarrollo Sostenible (ODS)~\cite{ODS} definidos por Naciones Unidas en 2015. Dichos objetivos buscan erradicar la pobreza, proteger el planeta y asegurar la prosperidad para todos como parte de una agenda global con metas a alcanzar en 2030. A continuación, se detalla cómo este proyecto contribuye a algunos de los ODS seleccionados.

\section{Objetivo número 4: Educación de calidad}
El desarrollo de esta aplicación web fomenta el aprendizaje y la accesibilidad de información técnica para el análisis de datos geográficos y temporales. Esto permite a estudiantes, investigadores y profesionales acceder a herramientas que antes podrían haber requerido conocimientos avanzados o recursos costosos. La democratización del acceso a estas herramientas fomenta una educación más inclusiva y accesible.

\subsection{Impacto en la práctica}
\begin{itemize}
    \item Formación técnica: Proporciona una plataforma para que los usuarios desarrollen habilidades prácticas en análisis de datos y visualización geoespacial.
    \item Inclusión educativa: Facilita el acceso a tecnologías avanzadas para comunidades con recursos limitados, ampliando las oportunidades de aprendizaje.
    \item Colaboración interdisciplinaria: Promueve el uso de tecnologías en áreas como la educación, la geografía y la sostenibilidad, fomentando proyectos colaborativos entre disciplinas.
\end{itemize}

\section{Objetivo número 9: Industria, innovación e infraestructura}
La creación de una aplicación optimizada para el análisis geoespacial y la gestión de datos temporales refuerza la importancia de la innovación en la digitalización de procesos tradicionales. Este proyecto promueve una infraestructura tecnológica accesible y eficiente, que puede ser replicada y adaptada para diversos contextos.

\subsection{Impacto en la práctica}
\begin{itemize}
    \item Promoción de herramientas sostenibles: La aplicación está diseñada para ser reutilizable y adaptable a distintos escenarios, minimizando la necesidad de crear nuevas herramientas desde cero.
    \item Innovación tecnológica: Utiliza técnicas avanzadas en visualización y análisis de datos para resolver problemas prácticos en diversas áreas de investigación.
    \item Eficiencia en la gestión de recursos: Al centralizar y procesar datos complejos en una sola plataforma, se reduce el tiempo y los recursos necesarios para realizar análisis.
\end{itemize}

\section{Objetivo número 13: Acción por el clima}
Aunque no de manera directa, este proyecto contribuye a la acción por el clima al facilitar la comprensión y el análisis de datos geoespaciales que pueden ser utilizados para estudios relacionados con el medio ambiente. Por ejemplo, permite analizar patrones de movilidad, cambios en el uso del suelo o distribución de eventos climáticos extremos, proporcionando información valiosa para la toma de decisiones sostenibles.

\subsection{Impacto en la práctica}
\begin{itemize}
    \item Análisis ambiental: Permite a los usuarios identificar patrones y tendencias en datos geográficos que pueden relacionarse con el cambio climático.
    \item Sensibilización: Al integrar datos geoespaciales con visualizaciones claras, ayuda a comunicar la urgencia de acciones relacionadas con el medio ambiente.
    \item Reducción de recursos: La herramienta digitaliza procesos que antes requerían análisis manuales o múltiples plataformas, disminuyendo el consumo de recursos materiales y energía.
\end{itemize}

\section{Conclusión}
Este proyecto no solo cumple con su propósito de ofrecer una herramienta eficiente para el análisis y visualización de datos geográficos, sino que también contribuye al avance de varios Objetivos de Desarrollo Sostenible. Al fomentar la educación inclusiva, la innovación tecnológica y la acción climática, este trabajo demuestra cómo pequeñas iniciativas pueden tener un impacto significativo en el logro de metas globales.



