\capitulo{2}{Objetivos del proyecto}

Este apartado explica de forma precisa y concisa cuales son los objetivos que se persiguen con la realización del proyecto. Se puede distinguir entre los objetivos marcados por los requisitos del software a construir y los objetivos de carácter técnico que plantea a la hora de llevar a la práctica el proyecto.

\section{Objetivos generales}\label{objetivos-generales}

\begin{itemize}
\tightlist
\item
  Desarrollar de un algoritmo de Big Data de nominado TRA-CLUS.
\item
  Facilitar la interpretación de los datos recogidos mediante
  representaciones gráficas.
\item
  Realizar comparativas de rendimiento contra diferentes algoritmos.
\item
  Desarrollar una aplicación web para mostrar los resultados.
\end{itemize}

\section{Objetivos técnicos}\label{objetivos-tecnicos}

\begin{itemize}
\tightlist
\item
  Desarrollar .
\item
  Desarrollar una aplicación () que con soporte...
\item
  Aplicar la arquitectura...
\item
  (Utilizar Gradle como herramienta para automatizar el proceso de
  construcción de software.)?
\item
  Hacer uso de herramientas ...
\item
  Aplicar la metodología ágil ...
\item
  (Realizar test unitarios, de integración y de interfaz.)?
\item
  Utilizar Git como sistema de control de versiones distribuido junto
  con la plataforma GitHub.
\item
  Utilizar GitHub Projects como herramienta de gestión de proyectos.
\end{itemize}

\section{Objetivos personales}\label{objetivos-personales}

\begin{itemize}
\tightlist
\item
  Realizar una aportación en el desarrollo experimental de software.
\item
  Reforzar conocimientos adquiridos durante la carrera.
\item
  Explorar metodologías y herramientas utilizadas en el entorno laboral.
\item
  Adentrarme en el campo del Big Data.
\item
  Profundizar en el desarrollo de aplicaciones ().
\end{itemize}
