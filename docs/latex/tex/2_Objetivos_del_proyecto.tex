\capitulo{2}{Objetivos del proyecto}

\section{Introducción}

Este apartado explica de forma precisa y concisa cuales son los objetivos que se persiguen con la realización del proyecto. Se puede distinguir entre los objetivos marcados por los requisitos del software a construir y los objetivos de carácter técnico que plantea a la hora de llevar a la práctica el proyecto.

\section{Objetivos generales}\label{objetivos-generales}

\begin{itemize}
    \item Implementar un algoritmo de Big Data denominado TRA-CLUS para el análisis y agrupación de trayectorias geoespaciales.
    \item Facilitar la interpretación de los datos recogidos empleando sistemas de GPS mediante representaciones gráficas claras y precisas.
    \item Realizar comparativas de rendimiento y precisión entre diferentes algoritmos de clustering aplicados en combinación con el algoritmo TRA-CLUS.
    \item Diseñar y desarrollar una aplicación web interactiva para mostrar y analizar los resultados obtenidos por el algoritmo.
\end{itemize}

\section{Objetivos técnicos}\label{objetivos-tecnicos}

\begin{itemize}
    \item Implementar el algoritmo TRA-CLUS optimizado para grandes volúmenes de datos utilizando herramientas eficientes y escalables.
    \item Diseñar y construir una aplicación web basada en \textit{Dash} que soporte la visualización de datos y permita la interacción del usuario.
    \item Aplicar una arquitectura modular y bien documentada que facilite el mantenimiento y la extensibilidad del proyecto.
%    \item Utilizar \textit{Gradle} o herramientas equivalentes para automatizar el proceso de construcción, pruebas y despliegue del software.
    \item Implementar herramientas de visualización como mapas interactivos para representar trayectorias y clusters.
    \item Aplicar metodologías ágiles como \textit{Scrum} para organizar y gestionar las etapas del desarrollo del proyecto.
    \item Realizar pruebas unitarias, de integración y de interfaz para garantizar la calidad y fiabilidad del sistema.
    \item Utilizar \textit{Git} como sistema de control de versiones distribuido y alojar el proyecto en la plataforma GitHub.
    \item Gestionar el desarrollo y el seguimiento del proyecto utilizando \textit{GitHub Projects} para organizar tareas y monitorear el progreso.
\end{itemize}

\section{Objetivos personales}\label{objetivos-personales}

\begin{itemize}
    \item Realizar una contribución significativa al desarrollo experimental de software aplicado al análisis de Big Data.
    \item Reforzar y consolidar los conocimientos adquiridos durante la carrera en el ámbito del desarrollo de software y análisis de datos.
    \item Explorar y aplicar metodologías y herramientas actuales utilizadas en entornos laborales profesionales.
    \item Adentrarme en el campo del Big Data y el análisis geoespacial para comprender mejor sus desafíos y oportunidades.
    \item Profundizar en el desarrollo de aplicaciones web modernas que integren múltiples componentes tecnológicos.
    \item Mejorar habilidades en gestión de proyectos, trabajo colaborativo y resolución de problemas técnicos complejos.
\end{itemize}

