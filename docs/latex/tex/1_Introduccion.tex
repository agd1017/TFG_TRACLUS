\capitulo{1}{Introducción}

El presente trabajo aborda el desarrollo e implementación de un sistema orientado al análisis de grandes volúmenes de datos geoespaciales mediante técnicas de agrupamiento o \textit{clustering}. En concreto, se ha centrado en el algoritmo TRA-CLUS, diseñado para segmentar y agrupar trayectorias geográficas, y en la creación de una aplicación web interactiva que facilite la interpretación y comparación de los resultados obtenidos con diferentes algoritmos de clustering.

El análisis de trayectorias es una tarea fundamental en campos como la movilidad urbana, la gestión del tráfico y la planificación territorial. Sin embargo, la complejidad y el volumen de datos asociados a estas áreas hacen necesario el uso de algoritmos eficientes y herramientas visuales que permitan extraer información útil de manera intuitiva. Este proyecto tiene como objetivo ofrecer una solución integral que combine una implementación optimizada del algoritmo TRA-CLUS con una interfaz visual de fácil uso.

La estructura de esta memoria está diseñada para presentar de manera clara y ordenada el desarrollo del proyecto:

\begin{itemize}
    \item \textbf{Capítulo 1 - Introducción}: Introduce el contexto general del proyecto, sus motivaciones y la organización de esta memoria.
    \item \textbf{Capítulo 2 - Objetivos del Proyecto}: Detalla los objetivos generales, técnicos y personales que han guiado el desarrollo del trabajo.
    \item \textbf{Capítulo 3 - Conceptos Teóricos}: Expone los fundamentos teóricos relacionados con el análisis de trayectorias, el clustering y las técnicas utilizadas en el desarrollo del proyecto.
    \item \textbf{Capítulo 4 - Técnicas y Herramientas}: Describe las tecnologías y herramientas empleadas, justificando su elección y explicando su aplicación dentro del proyecto.
    \item \textbf{Capítulo 5 - Aspectos Relevantes del Proyecto}: Presenta las decisiones clave tomadas durante el desarrollo, los desafíos enfrentados y las soluciones implementadas.
    \item \textbf{Capítulo 6 - Trabajos Relacionados}: Revisa proyectos y estudios previos que guardan relación con este trabajo, destacando sus aportaciones y diferencias con el presente proyecto.
    \item \textbf{Capítulo 7 - Conclusiones y Líneas de Trabajo Futuras}: Resume los resultados alcanzados, las lecciones aprendidas y propone posibles mejoras y extensiones para el proyecto.
\end{itemize}

En conclusión, este proyecto pretende no solo resolver un problema específico del análisis de trayectorias, sino también sentar las bases para futuras investigaciones y desarrollos en el ámbito del Big Data aplicado a datos geoespaciales.
