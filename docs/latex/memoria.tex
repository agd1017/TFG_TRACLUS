\documentclass[a4paper,12pt,twoside]{memoir}

% Castellano
\usepackage[spanish,es-tabla]{babel}
\selectlanguage{spanish}
\usepackage[utf8]{inputenc}
\usepackage[T1]{fontenc}
\usepackage{lmodern} % Scalable font
\usepackage{microtype}
\usepackage{placeins}

\RequirePackage{booktabs}
\RequirePackage[table]{xcolor}
\RequirePackage{xtab}
\RequirePackage{multirow}

% Links
\PassOptionsToPackage{hyphens}{url}\usepackage[colorlinks]{hyperref}
\hypersetup{
	allcolors = {red}
}

% Ecuaciones
\usepackage{amsmath}

% Rutas de fichero / paquete
\newcommand{\ruta}[1]{{\sffamily #1}}

% Párrafos
\nonzeroparskip

% Huérfanas y viudas
\widowpenalty100000
\clubpenalty100000

% Imágenes

% Comando para insertar una imagen en un lugar concreto.
% Los parámetros son:
% 1 --> Ruta absoluta/relativa de la figura
% 2 --> Texto a pie de figura
% 3 --> Tamaño en tanto por uno relativo al ancho de página
\usepackage{graphicx}
\newcommand{\imagen}[3]{
	\begin{figure}[!h]
		\centering
		\includegraphics[width=#3\textwidth]{#1}
		\caption{#2}\label{fig:#1}
	\end{figure}
	\FloatBarrier
}

% Comando para insertar una imagen sin posición.
% Los parámetros son:
% 1 --> Ruta absoluta/relativa de la figura
% 2 --> Texto a pie de figura
% 3 --> Tamaño en tanto por uno relativo al ancho de página
\newcommand{\imagenflotante}[3]{
	\begin{figure}
		\centering
		\includegraphics[width=#3\textwidth]{#1}
		\caption{#2}\label{fig:#1}
	\end{figure}
}

% El comando \figura nos permite insertar figuras comodamente, y utilizando
% siempre el mismo formato. Los parametros son:
% 1 --> Porcentaje del ancho de página que ocupará la figura (de 0 a 1)
% 2 --> Fichero de la imagen
% 3 --> Texto a pie de imagen
% 4 --> Etiqueta (label) para referencias
% 5 --> Opciones que queramos pasarle al \includegraphics
% 6 --> Opciones de posicionamiento a pasarle a \begin{figure}
\newcommand{\figuraConPosicion}[6]{%
  \setlength{\anchoFloat}{#1\textwidth}%
  \addtolength{\anchoFloat}{-4\fboxsep}%
  \setlength{\anchoFigura}{\anchoFloat}%
  \begin{figure}[#6]
    \begin{center}%
      \Ovalbox{%
        \begin{minipage}{\anchoFloat}%
          \begin{center}%
            \includegraphics[width=\anchoFigura,#5]{#2}%
            \caption{#3}%
            \label{#4}%
          \end{center}%
        \end{minipage}
      }%
    \end{center}%
  \end{figure}%
}

%
% Comando para incluir imágenes en formato apaisado (sin marco).
\newcommand{\figuraApaisadaSinMarco}[5]{%
  \begin{figure}%
    \begin{center}%
    \includegraphics[angle=90,height=#1\textheight,#5]{#2}%
    \caption{#3}%
    \label{#4}%
    \end{center}%
  \end{figure}%
}
% Para las tablas
\newcommand{\otoprule}{\midrule [\heavyrulewidth]}
%
% Nuevo comando para tablas pequeñas (menos de una página).
\newcommand{\tablaSmall}[5]{%
 \begin{table}
  \begin{center}
   \rowcolors {2}{gray!35}{}
   \begin{tabular}{#2}
    \toprule
    #4
    \otoprule
    #5
    \bottomrule
   \end{tabular}
   \caption{#1}
   \label{tabla:#3}
  \end{center}
 \end{table}
}

%
% Nuevo comando para tablas pequeñas (menos de una página).
\newcommand{\tablaSmallSinColores}[5]{%
 \begin{table}[H]
  \begin{center}
   \begin{tabular}{#2}
    \toprule
    #4
    \otoprule
    #5
    \bottomrule
   \end{tabular}
   \caption{#1}
   \label{tabla:#3}
  \end{center}
 \end{table}
}

\newcommand{\tablaApaisadaSmall}[5]{%
\begin{landscape}
  \begin{table}
   \begin{center}
    \rowcolors {2}{gray!35}{}
    \begin{tabular}{#2}
     \toprule
     #4
     \otoprule
     #5
     \bottomrule
    \end{tabular}
    \caption{#1}
    \label{tabla:#3}
   \end{center}
  \end{table}
\end{landscape}
}

%
% Nuevo comando para tablas grandes con cabecera y filas alternas coloreadas en gris.
\newcommand{\tabla}[6]{%
  \begin{center}
    \tablefirsthead{
      \toprule
      #5
      \otoprule
    }
    \tablehead{
      \multicolumn{#3}{l}{\small\sl continúa desde la página anterior}\\
      \toprule
      #5
      \otoprule
    }
    \tabletail{
      \hline
      \multicolumn{#3}{r}{\small\sl continúa en la página siguiente}\\
    }
    \tablelasttail{
      \hline
    }
    \bottomcaption{#1}
    \rowcolors {2}{gray!35}{}
    \begin{xtabular}{#2}
      #6
      \bottomrule
    \end{xtabular}
    \label{tabla:#4}
  \end{center}
}

%
% Nuevo comando para tablas grandes con cabecera.
\newcommand{\tablaSinColores}[6]{%
  \begin{center}
    \tablefirsthead{
      \toprule
      #5
      \otoprule
    }
    \tablehead{
      \multicolumn{#3}{l}{\small\sl continúa desde la página anterior}\\
      \toprule
      #5
      \otoprule
    }
    \tabletail{
      \hline
      \multicolumn{#3}{r}{\small\sl continúa en la página siguiente}\\
    }
    \tablelasttail{
      \hline
    }
    \bottomcaption{#1}
    \begin{xtabular}{#2}
      #6
      \bottomrule
    \end{xtabular}
    \label{tabla:#4}
  \end{center}
}

%
% Nuevo comando para tablas grandes sin cabecera.
\newcommand{\tablaSinCabecera}[5]{%
  \begin{center}
    \tablefirsthead{
      \toprule
    }
    \tablehead{
      \multicolumn{#3}{l}{\small\sl continúa desde la página anterior}\\
      \hline
    }
    \tabletail{
      \hline
      \multicolumn{#3}{r}{\small\sl continúa en la página siguiente}\\
    }
    \tablelasttail{
      \hline
    }
    \bottomcaption{#1}
  \begin{xtabular}{#2}
    #5
   \bottomrule
  \end{xtabular}
  \label{tabla:#4}
  \end{center}
}



\definecolor{cgoLight}{HTML}{EEEEEE}
\definecolor{cgoExtralight}{HTML}{FFFFFF}

%
% Nuevo comando para tablas grandes sin cabecera.
\newcommand{\tablaSinCabeceraConBandas}[5]{%
  \begin{center}
    \tablefirsthead{
      \toprule
    }
    \tablehead{
      \multicolumn{#3}{l}{\small\sl continúa desde la página anterior}\\
      \hline
    }
    \tabletail{
      \hline
      \multicolumn{#3}{r}{\small\sl continúa en la página siguiente}\\
    }
    \tablelasttail{
      \hline
    }
    \bottomcaption{#1}
    \rowcolors[]{1}{cgoExtralight}{cgoLight}

  \begin{xtabular}{#2}
    #5
   \bottomrule
  \end{xtabular}
  \label{tabla:#4}
  \end{center}
}



\graphicspath{ {./img/} }

% Capítulos
\chapterstyle{bianchi}
\newcommand{\capitulo}[2]{
	\setcounter{chapter}{#1}
	\setcounter{section}{0}
	\setcounter{figure}{0}
	\setcounter{table}{0}
	\chapter*{\thechapter.\enskip #2}
	\addcontentsline{toc}{chapter}{\thechapter.\enskip #2}
	\markboth{#2}{#2}
}

% Apéndices
\renewcommand{\appendixname}{Apéndice}
\renewcommand*\cftappendixname{\appendixname}

\newcommand{\apendice}[1]{
	%\renewcommand{\thechapter}{A}
	\chapter{#1}
}

\renewcommand*\cftappendixname{\appendixname\ }

% Formato de portada
\makeatletter
\usepackage{xcolor}
\newcommand{\tutor}[1]{\def\@tutor{#1}}
\newcommand{\course}[1]{\def\@course{#1}}
\definecolor{cpardoBox}{HTML}{E6E6FF}
\def\maketitle{
  \null
  \thispagestyle{empty}
  % Cabecera ----------------
\noindent\includegraphics[width=\textwidth]{cabecera}\vspace{1cm}%
  \vfill
  % Título proyecto y escudo informática ----------------
  \colorbox{cpardoBox}{%
    \begin{minipage}{.8\textwidth}
      \vspace{.5cm}\Large
      \begin{center}
      \textbf{TFG del Grado en Ingeniería Informática}\vspace{.6cm}\\
      \textbf{\LARGE\@title{}}
      \end{center}
      \vspace{.2cm}
    \end{minipage}

  }%
  \hfill\begin{minipage}{.20\textwidth}
    \includegraphics[width=\textwidth]{escudoInfor}
  \end{minipage}
  \vfill
  % Datos de alumno, curso y tutores ------------------
  \begin{center}%
  {%
    \noindent\LARGE
    Presentado por \@author{}\\ 
    en Universidad de Burgos --- \@date{}\\
    Tutor: \@tutor{}\\
  }%
  \end{center}%
  \null
  \cleardoublepage
  }
\makeatother

\newcommand{\nombre}{Álvaro González Delgado} %%% cambio de comando

% Datos de portada
\title{Implementación del algoritmo TRA-CLUS Mejorado para el análisis de tráfico}
\author{\nombre}
\tutor{Bruno Baruque Zanon y Hector Cogollos Adrian}
\date{\today}

\begin{document}

\maketitle


\newpage\null\thispagestyle{empty}\newpage


%%%%%%%%%%%%%%%%%%%%%%%%%%%%%%%%%%%%%%%%%%%%%%%%%%%%%%%%%%%%%%%%%%%%%%%%%%%%%%%%%%%%%%%%
\thispagestyle{empty}


\noindent\includegraphics[width=\textwidth]{cabecera}\vspace{1cm}

\noindent D. nombre tutor, profesor del departamento de nombre departamento, área de nombre área.

\noindent Expone:

\noindent Que el alumno D. \nombre, con DNI 71482750S, ha realizado el Trabajo final de Grado en Ingeniería Informática titulado título de TFG. 

\noindent Y que dicho trabajo ha sido realizado por el alumno bajo la dirección del que suscribe, en virtud de lo cual se autoriza su presentación y defensa.

\begin{center} %\large
En Burgos, {\large \today}
\end{center}

\vfill\vfill\vfill

% Author and supervisor
\begin{minipage}{0.45\textwidth}
\begin{flushleft} %\large
Vº. Bº. del Tutor:\\[2cm]
D. Bruno Baruque Zanon
\end{flushleft}
\end{minipage}
\hfill
\begin{minipage}{0.45\textwidth}
\begin{flushleft} %\large
Vº. Bº. del co-tutor:\\[2cm]
D. Hector Cogollos Adrian
\end{flushleft}
\end{minipage}
\hfill

\vfill

% para casos con solo un tutor comentar lo anterior
% y descomentar lo siguiente
%Vº. Bº. del Tutor:\\[2cm]
%D. nombre tutor


\newpage\null\thispagestyle{empty}\newpage




\frontmatter

% Abstract en castellano
\renewcommand*\abstractname{Resumen}
\begin{abstract}
El presente proyecto tiene como objetivo el desarrollo de una aplicación web interactiva para la visualización y análisis de trayectorias mediante la implementación del algoritmo de agrupamiento de trayectorias TRACLUS y el estudio de sus variantes. La solución integra herramientas de procesamiento de datos y visualización para facilitar la interpretación de patrones espaciales en trayectorias, con énfasis en eficiencia y aplicabilidad en entornos de Big Data. Se emplearon metodologías ágiles y tecnologías modernas como Dash y scikit-learn para garantizar un diseño modular y adaptable.
\end{abstract}

\renewcommand*\abstractname{Descriptores}
\begin{abstract}
TRACLUS, análisis de trayectorias, agrupamiento, visualización de datos, Dash, scikit-learn, Big Data.
\end{abstract}

\clearpage

% Abstract en inglés
\renewcommand*\abstractname{Abstract}
\begin{abstract}
This project aims to develop an interactive web application for the visualization and analysis of trajectories through the implementation of the TRACLUS trajectory clustering algorithm and the study of its variants. The solution integrates data processing and visualization tools to facilitate the interpretation of spatial patterns in trajectories, focusing on efficiency and applicability in Big Data environments. Agile methodologies and modern technologies such as Dash and scikit-learn were employed to ensure a modular and adaptable design.
\end{abstract}

\renewcommand*\abstractname{Keywords}
\begin{abstract}
TRACLUS, trajectory analysis, clustering, data visualization, Dash, scikit-learn, Big Data.
\end{abstract}


\clearpage

% Indices
\tableofcontents

\clearpage

\listoffigures

\clearpage

\listoftables
\clearpage

\mainmatter
\capitulo{1}{Introducción}

El presente trabajo aborda el desarrollo e implementación de un sistema orientado al análisis de grandes volúmenes de datos geoespaciales mediante técnicas de agrupamiento o \textit{clustering}. En concreto, se ha centrado en el algoritmo TRA-CLUS, diseñado para segmentar y agrupar trayectorias GPS, y en la creación de una aplicación web interactiva que facilite la interpretación y comparación de los resultados obtenidos con diferentes algoritmos de clustering.

El análisis de trayectorias es una tarea fundamental en campos como la movilidad urbana, la gestión del tráfico y la planificación territorial. Sin embargo, la complejidad y el volumen de datos asociados a estas áreas hacen necesario el uso de algoritmos eficientes y herramientas visuales que permitan extraer información útil de manera intuitiva. Este proyecto tiene como objetivo ofrecer una solución integral que combine una implementación optimizada del algoritmo TRA-CLUS con una interfaz visual de fácil uso.

La estructura de esta memoria está diseñada para presentar de manera clara y ordenada el desarrollo del proyecto:

\begin{itemize}
    \item \textbf{Capítulo 1 - Introducción}: Introduce el contexto general del proyecto, sus motivaciones y la organización de esta memoria.
    \item \textbf{Capítulo 2 - Objetivos del Proyecto}: Detalla los objetivos generales, técnicos y personales que han guiado el desarrollo del trabajo.
    \item \textbf{Capítulo 3 - Conceptos Teóricos}: Expone los fundamentos teóricos relacionados con el análisis de trayectorias, el clustering y las técnicas utilizadas en el desarrollo del proyecto.
    \item \textbf{Capítulo 4 - Técnicas y Herramientas}: Describe las tecnologías y herramientas empleadas, justificando su elección y explicando su aplicación dentro del proyecto.
    \item \textbf{Capítulo 5 - Aspectos Relevantes del Proyecto}: Presenta las decisiones clave tomadas durante el desarrollo, los desafíos enfrentados y las soluciones implementadas.
    \item \textbf{Capítulo 6 - Trabajos Relacionados}: Revisa proyectos y estudios previos que guardan relación con este trabajo, destacando sus aportaciones y diferencias con el presente proyecto.
    \item \textbf{Capítulo 7 - Conclusiones y Líneas de Trabajo Futuras}: Resume los resultados alcanzados, las lecciones aprendidas y propone posibles mejoras y extensiones para el proyecto.
\end{itemize}

En conclusión, este proyecto pretende no solo resolver un problema específico del análisis de trayectorias, sino también sentar las bases para futuras investigaciones y desarrollos en el ámbito del Big Data aplicado a datos geoespaciales.

\capitulo{2}{Objetivos del proyecto}

Este apartado explica de forma precisa y concisa cuales son los objetivos que se persiguen con la realización del proyecto. Se puede distinguir entre los objetivos marcados por los requisitos del software a construir y los objetivos de carácter técnico que plantea a la hora de llevar a la práctica el proyecto.

\section{Objetivos generales}\label{objetivos-generales}

\begin{itemize}
\tightlist
\item
  Desarrollar de un algoritmo de Big Data de nominado TRA-CLUS.
\item
  Facilitar la interpretación de los datos recogidos mediante
  representaciones gráficas.
\item
  Realizar comparativas de rendimiento contra diferentes algoritmos.
\item
  Desarrollar una aplicación (web o escritorio) para mostrar los resultados.
\end{itemize}

\section{Objetivos técnicos}\label{objetivos-tecnicos}

\begin{itemize}
\tightlist
\item
  Desarrollar .
\item
  Desarrollar una aplicación () que con soporte...
\item
  Aplicar la arquitectura...
\item
  (Utilizar Gradle como herramienta para automatizar el proceso de
  construcción de software.)?
\item
  Hacer uso de herramientas ...
\item
  Aplicar la metodología ágil ...
\item
  (Realizar test unitarios, de integración y de interfaz.)?
\item
  Utilizar Git como sistema de control de versiones distribuido junto
  con la plataforma GitHub.
\item
  Utilizar GitHub Projects como herramienta de gestión de proyectos.
\end{itemize}

\section{Objetivos personales}\label{objetivos-personales}

\begin{itemize}
\tightlist
\item
  Realizar una aportación en el desarrollo experimental de software.
\item
  Reforzar conocimientos adquiridos durante la carrera.
\item
  Explorar metodologías y herramientas utilizadas en el entorno laboral.
\item
  Adentrarme en el campo del Big Data.
\item
  Profundizar en el desarrollo de aplicaciones ().
\end{itemize}

\capitulo{3}{Conceptos teóricos}

En este apartado se presentan los conceptos teóricos fundamentales que permiten comprender el marco conceptual en el que se desarrolla este trabajo. Estos conceptos proporcionan el contexto necesario para el análisis y desarrollo del estudio realizado.

La discusión se centrará en los principios relacionados con el algoritmo TRACLUS, dado que este ha sido el enfoque principal del estudio y representa la mayor complejidad en su implementación y análisis.

\section{Trayectorias GPS}

Las trayectorias GPS se refieren a secuencias de puntos geoespaciales capturados mediante dispositivos de posicionamiento global (GPS). Cada punto de una trayectoria contiene información geográfica, como la latitud, longitud, y, en algunos casos, la altitud y el tiempo asociado. Estas trayectorias son fundamentales para analizar el movimiento de objetos o individuos a lo largo del tiempo y el espacio.

\begin{itemize}
    \item \textbf{Formato típico}: Una trayectoria GPS suele representarse como una lista ordenada de coordenadas \([x, y]\), donde \(x\) corresponde a la longitud y \(y\) a la latitud. Un ejemplo típico es:
    \[
    [[\text{longitud}_1, \text{latitud}_1], [\text{longitud}_2, \text{latitud}_2], \dots]
    \]
    \item \textbf{Origen de los datos}: Estas trayectorias se generan a partir de dispositivos móviles, vehículos, sensores de navegación y otros sistemas de rastreo que capturan posiciones geográficas en intervalos regulares de tiempo.
    \item \textbf{Aplicaciones}: Las trayectorias GPS son esenciales en áreas como la planificación de rutas, el análisis del tráfico, el monitoreo ambiental y la movilidad urbana. También sirven como base para algoritmos de agrupamiento y análisis de datos geoespaciales, como el algoritmo TRA-CLUS.
\end{itemize}

La importancia de las trayectorias GPS radica en su capacidad para modelar patrones de movimiento complejos, facilitando la identificación de tendencias, comportamientos y anomalías en contextos espaciales y temporales. Sin embargo, su análisis presenta desafíos debido a la densidad y complejidad de los datos.


\section{Clustering}

El \textbf{clustering} o agrupamiento es una técnica de aprendizaje no supervisado utilizada para organizar datos en grupos o "clusters" basados en características similares. Cada \textit{cluster} está compuesto por elementos más similares entre sí que a elementos de otros \textit{clusters}. Esta técnica es esencial en análisis exploratorio, permitiendo descubrir estructuras subyacentes en grandes volúmenes de datos y encontrar patrones, sin necesidad de tener etiquetas o categorías predefinidas.

En el contexto de análisis de datos, el clustering se aplica en múltiples áreas como la segmentación de clientes, detección de patrones de comportamiento, agrupación de imágenes y análisis de redes sociales. Entre los métodos de clustering más utilizados en la práctica se incluyen algoritmos basados en densidad y en conectividad, que proporcionan flexibilidad y adaptabilidad para manejar datos complejos y de alta dimensionalidad.

\subsection*{Clustering en el TRA-CLUS}

En el algoritmo TRA-CLUS, el proceso de agrupamiento o \textit{clustering} es una parte fundamental que se utiliza para identificar trayectorias similares basándose en los segmentos generados tras la partición de las trayectorias originales. Para este proyecto, se considera la posibilidad de sustituir el método de agrupamiento original de TRA-CLUS por algoritmos más avanzados o específicos que permitan optimizar los resultados.

La elección de un algoritmo de clustering para TRA-CLUS debe cumplir con las siguientes condiciones clave:
\begin{itemize}
    \item \textbf{Métrica de Distancia Predefinida}: El algoritmo debe ser compatible con una métrica de similitud o distancia previamente calculada, es decir, debe aceptar matrices de distancias precomputadas. Esto es esencial, ya que la similitud entre segmentos en TRA-CLUS se mide previamente y se organiza en una matriz de distancias.
    \item \textbf{Adaptación a Densidades Variables}: Dado que las trayectorias suelen estar distribuidas en áreas con densidades variables, el algoritmo debe ser capaz de manejar estas diferencias para evitar sesgos en los resultados de agrupamiento.
    \item \textbf{Eficiencia Computacional}: Dado el volumen potencialmente grande de datos en estudios de trayectorias, el algoritmo debe ser computacionalmente eficiente para garantizar tiempos de procesamiento razonables.
\end{itemize}

\subsection*{Algoritmos de Clustering en scikit-learn}

A continuación, se presenta una descripción de los algoritmos de clustering que cumplen las condiciones anteriores de la biblioteca \texttt{scikit-learn} de Python:

\subsubsection*{1. DBSCAN (Density-Based Spatial Clustering of Applications with Noise)}

DBSCAN es un algoritmo basado en densidad que agrupa puntos que están en áreas de alta densidad y considera como ruido aquellos que se encuentran en áreas de baja densidad. Los clusters se forman alrededor de puntos densamente conectados y son identificados por dos parámetros: el radio de vecindad (\(\epsilon\)) y el número mínimo de puntos necesarios (\textit{minPts}) para formar un cluster. DBSCAN es especialmente útil para datos con formas irregulares y ruido, ya que ignora puntos aislados que no pertenecen a ninguna agrupación significativa.

\subsubsection*{2. OPTICS (Ordering Points To Identify the Clustering Structure)}

OPTICS es una extensión de DBSCAN que aborda el problema de la sensibilidad a la elección de \(\epsilon\). En lugar de identificar clusters individuales directamente, OPTICS produce una ordenación de los puntos que muestra su estructura de densidad subyacente. Esto permite descubrir clusters a múltiples escalas y niveles de densidad, haciendo posible una mayor flexibilidad en la agrupación de datos.

\subsubsection*{3. HDBSCAN (Hierarchical Density-Based Spatial Clustering of Applications with Noise)}

HDBSCAN es una variante jerárquica de DBSCAN que forma clusters de densidad utilizando una estructura jerárquica en lugar de depender de un valor fijo de \(\epsilon\). A diferencia de DBSCAN y OPTICS, HDBSCAN construye una jerarquía de clusters que permite identificar agrupaciones en diferentes niveles de densidad sin requerir parámetros estrictos. Este algoritmo es particularmente útil cuando la densidad de los clusters varía significativamente.

\subsubsection*{4. Spectral Clustering}

Spectral Clustering es un algoritmo de agrupación basado en teoría de grafos y técnicas de álgebra lineal. Utiliza los valores propios de una matriz de similitud de los datos para realizar la agrupación. Este enfoque es particularmente adecuado para datos que presentan estructuras de clusters no lineales o formas complejas. Spectral Clustering convierte el problema de agrupación en uno de corte de grafos, dividiendo el conjunto de datos en \textit{k} clusters minimizando la similitud entre los clusters.

\subsubsection*{5. Agglomerative Clustering}

El \textbf{Agglomerative Clustering} es una técnica jerárquica de clustering donde cada punto comienza como su propio cluster, y los clusters se fusionan iterativamente en función de una métrica de distancia (como la distancia euclidiana, de Manhattan, o de enlace promedio) hasta que se alcanza el número deseado de clusters o se completa la jerarquía. Este método es particularmente útil cuando se requiere una representación visual de los clusters en forma de dendrograma.

\subsection*{Comparación de los Algoritmos de Clustering}

Los algoritmos de clustering mencionados se diferencian principalmente en su enfoque de agrupación (por densidad, jerárquico o basado en similitud), su sensibilidad a la elección de parámetros y su capacidad para manejar clusters de diferentes formas y densidades. A continuación, se presenta una tabla comparativa de los algoritmos:

\begin{table}[ht]
\centering
\begin{tabular}{|p{2.5cm}|p{3.7cm}|p{3.7cm}|p{4cm}|}
\hline
\textbf{Algoritmo} & \textbf{Tipo de Clustering} & \textbf{Ventaja Principal} & \textbf{Limitación Principal} \\
\hline
DBSCAN & Densidad & Maneja ruido y clusters de formas arbitrarias & Sensible a la elección de \(\epsilon\) y \textit{minPts} \\
OPTICS & Densidad & Detecta clusters a diferentes niveles de densidad & Complejo de interpretar \\
HDBSCAN & Jerárquico basado en densidad & Sin parámetros estrictos & Computacionalmente costoso \\
Spectral Clustering & Basado en grafos & Captura estructuras complejas & Requiere especificar \textit{k} \\
Agglomerative Clustering & Jerárquico & Ofrece dendrograma jerárquico & Alta complejidad para datos grandes \\
\hline
\end{tabular}
\end{table}

\subsubsection*{Conclusión}

Los algoritmos basados en densidad, como DBSCAN, OPTICS y HDBSCAN, son especialmente útiles para datos con clusters de formas irregulares y en presencia de ruido, aunque requieren ajustes específicos de parámetros o presentan alta complejidad computacional. Por otro lado, métodos como Spectral Clustering y Agglomerative Clustering son más adecuados para datos que exhiben estructuras jerárquicas o formas no lineales, pero a costa de una mayor sensibilidad a la configuración inicial y un mayor costo computacional en datasets grandes.

En el contexto del algoritmo TRA-CLUS, donde la similitud entre segmentos se precomputa y los datos presentan densidades variables, los algoritmos basados en densidad, como HDBSCAN, ofrecen ventajas significativas por su flexibilidad y capacidad para manejar variaciones en los datos sin requerir parámetros estrictos. Sin embargo, la elección del algoritmo debe balancear la precisión de los resultados con la eficiencia computacional, considerando las características específicas de las trayectorias a analizar.


\section{TRACLUS}

TRACLUS \cite{lee2007trajectory} es un algoritmo de agrupación (\textit{clustering}) especializado en datos de trayectorias, diseñado para identificar patrones comunes de movimiento en conjuntos de datos de trayectorias, como rutas de vehículos, movimientos de animales o trayectorias de fenómenos meteorológicos. A diferencia de los algoritmos tradicionales de \textit{clustering} que agrupan puntos individuales, TRACLUS se enfoca en el agrupamiento de trayectorias completas, descomponiendo cada trayectoria en segmentos y detectando patrones comunes en subtrayectorias específicas. Este enfoque es útil en estudios donde los objetos presentan secuencias de movimiento en el espacio y el tiempo, permitiendo identificar similitudes parciales dentro de grandes volúmenes de datos.

\subsection*{Principios y Funcionamiento de TRACLUS}

El funcionamiento de TRACLUS se basa en dos etapas principales: la segmentación de trayectorias y la agrupación de segmentos de trayectorias. Ambas etapas están diseñadas para abordar la naturaleza secuencial y direccional de las trayectorias, empleando un enfoque basado en densidad que permite una identificación precisa de subtrayectorias similares.

\begin{enumerate}
    \item \textbf{Segmentación de Trayectorias}: La primera etapa de TRACLUS es dividir cada trayectoria en segmentos de línea más cortos en función de cambios direccionales o puntos de inflexión. Estos puntos característicos dividen la trayectoria en subtrayectorias que pueden ser más fácilmente comparables. Este paso es crucial porque permite detectar patrones comunes en segmentos específicos, en lugar de requerir una coincidencia exacta en toda la trayectoria.

    \item \textbf{Agrupación Basada en Densidad de Segmentos}: En lugar de agrupar puntos aislados, TRACLUS agrupa segmentos que se encuentran en regiones densas del espacio de trayectoria mediante una adaptación del algoritmo DBSCAN (\textit{Density-Based Spatial Clustering of Applications with Noise}). Esta agrupación basada en densidad identifica áreas de alta concentración de segmentos similares que constituyen patrones de movimiento comunes. Los clusters se forman en áreas de alta densidad de segmentos, separadas por regiones de baja densidad, permitiendo agrupar segmentos que comparten características similares, incluso si otras partes de la trayectoria son diferentes.
\end{enumerate}

\subsection*{Métrica de Similitud en TRACLUS}

TRACLUS emplea una métrica de similitud diseñada específicamente para medir la relación entre segmentos individuales, en lugar de comparar trayectorias completas. Esta métrica evalúa la similitud entre segmentos considerando tres aspectos principales:

\begin{itemize}
    \item \textbf{Distancia perpendicular}: Mide la distancia más corta desde un punto de un segmento hasta la línea definida por el otro segmento. Esta medida capta la proximidad entre los segmentos desde una perspectiva geométrica.
    \item \textbf{Distancia paralela}: Evalúa la proyección de un segmento sobre el otro, midiendo cuánto de un segmento está alineado en la misma dirección que el otro. Esto permite identificar segmentos con orientaciones similares.
    \item \textbf{Diferencia en la longitud}: Compara las longitudes de los segmentos, lo que ayuda a identificar similitudes en términos de escala o tamaño.
\end{itemize}

La métrica de similitud en TRACLUS combina estas tres distancias en una medida agregada, lo que permite capturar tanto la proximidad espacial como la alineación y el tamaño relativo entre segmentos. Este enfoque es ideal para analizar conjuntos de datos con trayectorias complejas, ya que facilita la identificación de patrones subyacentes al dividir trayectorias en segmentos más pequeños.

A diferencia de otras métricas como DTW (Dynamic Time Warping) o LCSS (Longest Common Subsequence), que se enfocan en medir la similitud entre series temporales completas, la métrica de TRACLUS proporciona mayor flexibilidad al detectar subtrayectorias similares. Esto resulta especialmente útil para aplicaciones que requieren identificar patrones locales dentro de trayectorias extensas y densas.

\subsection*{Ventajas de TRACLUS en el Análisis de Trayectorias}
El algoritmo TRACLUS ofrece varias ventajas que lo hacen adecuado para el análisis de datos de trayectoria:

\begin{itemize}
    \item \textbf{Identificación de Subtrayectorias Similares}: TRACLUS no se limita a identificar patrones en trayectorias completas, sino que permite detectar similitudes en segmentos específicos. Esta capacidad es esencial en contextos donde solo algunas secciones de las trayectorias son comparables, mientras que otras presentan variaciones.
    
    \item \textbf{Adaptación a Escalas Variables}: Aunque TRACLUS es sensible a la densidad debido a su enfoque basado en clustering por densidad, su capacidad para manejar variaciones en la escala de las trayectorias lo hace útil en aplicaciones heterogéneas, como estudios de tráfico o movimientos de fauna en diferentes ecosistemas.
    
    \item \textbf{Selección Automática de Parámetros}: Mediante el uso de heurísticas para definir automáticamente los valores de parámetros clave (como el radio de vecindad $\epsilon$ y el número mínimo de puntos vecinos), TRACLUS reduce la necesidad de ajustes manuales. Sin embargo, la elección adecuada de estos parámetros sigue siendo crucial para garantizar resultados precisos en diferentes conjuntos de datos.
\end{itemize}


\subsection*{Aplicaciones de TRACLUS en la Investigación}
TRACLUS se puede aplicar en múltiples campos de investigación, donde el análisis de patrones de movimiento es fundamental:

\begin{itemize}
    \item \textbf{Biología y Ecología}: Para analizar trayectorias de animales, identificando patrones de comportamiento, rutas migratorias o territorios de caza.
    
    \item \textbf{Meteorología}: Para el estudio de trayectorias de fenómenos climáticos como huracanes, permitiendo identificar patrones comunes en ciertos eventos.
    
    \item \textbf{Gestión del Tráfico y Transporte}: En el análisis de rutas vehiculares, detectando patrones de congestión, flujo de tráfico y rutas populares.
\end{itemize}



\capitulo{4}{Técnicas y herramientas}

En esta sección se presentan las técnicas metodológicas y las herramientas de desarrollo que se han utilizado para llevar a cabo el proyecto. Se han considerado diferentes alternativas de metodologías y herramientas, y se ofrece un resumen de los aspectos más destacados de cada opción, junto con una justificación de las elecciones realizadas.

\section{GitHub}
GitHub es una plataforma de desarrollo colaborativo basada en la web que utiliza el sistema de control de versiones Git. Es ampliamente utilizado en la comunidad de desarrollo de software por varias razones:

\begin{itemize}
    \item \textbf{Control de versiones:} GitHub facilita el seguimiento de cambios en el código, lo que permite a los desarrolladores revertir a versiones anteriores si es necesario. Esto es esencial para la gestión de errores y la mejora continua del software.
    \item \textbf{Integración continua:} GitHub se puede integrar con diversas herramientas de automatización, como GitHub Actions, que facilitan la construcción, prueba y despliegue automático del código, mejorando la eficiencia del flujo de trabajo.
    \item \textbf{Documentación y seguimiento de issues:} Proporciona herramientas para documentar el código y gestionar tareas o problemas a través de un sistema de “issues”, lo que facilita la organización y planificación del desarrollo.
\end{itemize}

\section{Texmaker}
Texmaker es un editor de texto multiplataforma para la creación de documentos en \LaTeX. Es una herramienta esencial para la redacción académica y técnica, y ha sido fundamental en la redacción de este trabajo por las siguientes razones:

\begin{itemize}
    \item \textbf{Interfaz amigable:} Proporciona un entorno intuitivo y fácil de usar que facilita la edición de documentos en \LaTeX, incluso para aquellos que son nuevos en el sistema.
    \item \textbf{Compilación rápida:} Permite compilar documentos \LaTeX rápidamente con un solo clic, lo que agiliza el proceso de revisión y mejora la productividad.
    \item \textbf{Herramientas integradas:} Incluye herramientas para la gestión de bibliografías, la inserción de gráficos y tablas, así como un visor PDF integrado que facilita la revisión del documento final.
    \item \textbf{Plantillas y ejemplos:} Ofrece diversas plantillas y ejemplos que ayudan a los usuarios a comenzar rápidamente con sus documentos, promoviendo buenas prácticas en la redacción científica.
\end{itemize}

\section{CSS}
Cascading Style Sheets (CSS) es un lenguaje utilizado para describir la presentación de documentos HTML y XML. En este proyecto, CSS se ha utilizado para mejorar la estética y la usabilidad de la interfaz de la aplicación desarrollada en Python con Dash. Las razones para su elección incluyen:

\begin{itemize}
    \item \textbf{Separación de contenido y estilo:} CSS permite mantener el contenido HTML separado de su presentación, lo que facilita el mantenimiento del código y la implementación de cambios en el diseño sin afectar el contenido.
    \item \textbf{Responsividad:} Facilita el diseño responsivo, asegurando que la aplicación se vea bien en diferentes dispositivos y tamaños de pantalla. Esto es crucial para mejorar la accesibilidad y la experiencia del usuario.
    \item \textbf{Personalización:} Proporciona flexibilidad para personalizar la apariencia de la aplicación de manera sencilla, permitiendo la creación de un diseño atractivo y funcional que se alinee con los objetivos del proyecto.
    \item \textbf{Compatibilidad:} CSS es compatible con todos los navegadores modernos, lo que asegura que el diseño se mantenga consistente en diferentes plataformas.
\end{itemize}

\section{Python}
Python es un lenguaje de programación versátil y fácil de aprender, ampliamente utilizado en el desarrollo de aplicaciones web, análisis de datos y machine learning. En este proyecto, se han utilizado varias bibliotecas de Python que han enriquecido el desarrollo y la funcionalidad de la aplicación:

\begin{itemize}
    \item \textbf{scikit-learn:} Una biblioteca para machine learning que ofrece herramientas eficientes para el análisis predictivo y la implementación de algoritmos de clustering, como DBSCAN y OPTICS. Su diseño optimizado permite realizar análisis complejos de manera rápida y sencilla.
 
    \item \textbf{Pandas:} Una biblioteca de análisis de datos que proporciona estructuras de datos y herramientas de manipulación para trabajar con datos tabulares. Permite la limpieza y transformación de datos, lo que es esencial para preparar los datos antes del análisis.
    \item \textbf{GeoPandas:} Extiende las capacidades de Pandas para trabajar con datos geoespaciales, permitiendo realizar análisis y visualizaciones de datos geográficos de forma eficiente. Esto es crucial para proyectos que requieren análisis de datos basados en ubicación.
    \item \textbf{Matplotlib:} Una biblioteca para crear visualizaciones estáticas, animadas e interactivas en Python. Es esencial para la representación gráfica de los resultados del análisis, facilitando la comunicación de hallazgos a través de gráficos claros y efectivos.
    \item \textbf{NumPy:} Una biblioteca fundamental para realizar cálculos numéricos en Python, que proporciona soporte para matrices y funciones matemáticas. NumPy es la base sobre la cual se construyen muchas otras bibliotecas de ciencia de datos y machine learning.
    \item \textbf{Shapely:} Utilizada para manipular y analizar geometrías en Python, es crucial en la gestión de datos geoespaciales, permitiendo realizar operaciones geométricas como intersecciones y uniones de formas.
    \item \textbf{JSON:} Una biblioteca que permite trabajar con datos en formato JSON, facilitando la interacción con APIs y el manejo de configuraciones. Esto es importante para integrar servicios externos en la aplicación.
    \item \textbf{Zipfile:} Utilizada para crear y leer archivos ZIP, facilitando la gestión de datos comprimidos y mejorando la eficiencia en la transferencia de datos.
    \item \textbf{Contextily:} Permite añadir mapas de fondo a las visualizaciones geográficas, mejorando la contextualización de los datos y ayudando a los usuarios a interpretar la información espacial.
    \item \textbf{io:} Proporciona funciones para manejar flujos de entrada y salida, útil en la manipulación de datos en memoria, permitiendo una gestión eficiente de archivos y datos temporales.
    \item \textbf{pyproj:} Una biblioteca para realizar transformaciones de coordenadas, crucial para el trabajo con datos geográficos y la interoperabilidad entre diferentes sistemas de referencia espacial.
    \item \textbf{Time y threading:} Utilizadas para gestionar la temporización y la ejecución de múltiples hilos de ejecución en la aplicación, lo que mejora la eficiencia y la capacidad de respuesta de la aplicación en tareas concurrentes.
    \item \textbf{Base64:} Facilita la codificación y decodificación de datos en formato Base64, útil para la transferencia de datos binarios, como imágenes o archivos, en formatos que requieren representación textual.
    \item \textbf{Dash:} 
    Dash es un marco de trabajo desarrollado por Plotly que permite la creación de aplicaciones web analíticas e interactivas utilizando Python. Es especialmente popular en la comunidad de ciencia de datos y visualización debido a sus características y beneficios:
    \begin{itemize}
        \item \textbf{Interactividad:} Dash permite crear aplicaciones que responden a las interacciones del usuario, como clics, desplazamientos y entradas de datos. Esto es crucial para el análisis de datos en tiempo real y la visualización interactiva.
        \item \textbf{Integración con Plotly:} Las visualizaciones de Dash se basan en la biblioteca Plotly, que permite crear gráficos complejos y visualizaciones de alta calidad con facilidad. Esto enriquece la presentación de datos y facilita la comunicación de resultados.
        \item \textbf{Composición de componentes:} Dash permite combinar diferentes componentes (gráficos, tablas, controles de entrada) en una sola interfaz, lo que facilita la creación de aplicaciones integrales que ofrecen una experiencia de usuario fluida.
        \item \textbf{Despliegue sencillo:} Las aplicaciones construidas con Dash se pueden desplegar fácilmente en servidores web, lo que permite compartir los resultados del análisis con un público más amplio sin requerir instalación adicional por parte del usuario final.
        \item \textbf{Flexibilidad:} Al estar basado en Python, los desarrolladores pueden aprovechar la amplia gama de bibliotecas disponibles para manipular datos, realizar análisis y crear visualizaciones personalizadas, lo que proporciona gran flexibilidad en el desarrollo de aplicaciones.
    \end{itemize}
\end{itemize}

Estas herramientas y bibliotecas han sido elegidas por su capacidad para facilitar el desarrollo, mejorar la eficiencia del trabajo y proporcionar funcionalidades que son fundamentales para el éxito del proyecto. Cada una de ellas contribuye a un enfoque integral que permite abordar las necesidades del análisis de datos y la creación de aplicaciones web interactivas.



\capitulo{5}{Aspectos relevantes del desarrollo del proyecto}

\section{Formación y aprendizaje necesario}

El desarrollo del proyecto requirió adquirir nuevos conocimientos y profundizar en diversas tecnologías y técnicas de análisis. A continuación, se describe la formación llevada a cabo en las herramientas y algoritmos más relevantes.

\subsection{Estudio del algoritmo TRACLUS}

El enfoque principal de este proyecto ha sido estudiar en profundidad el algoritmo \textbf{TRACLUS}, destinando una gran parte de las horas de investigación a comprender su funcionamiento y potencial para el análisis de trayectorias. Este algoritmo se basa en la segmentación y agrupación de trayectorias, permitiendo identificar sub-trayectorias comunes dentro de un conjunto de datos. Esto lo convierte en una herramienta valiosa para descubrir patrones significativos en datos de trayectorias. La investigación incluyó la revisión de artículos académicos y la exploración de cómo ajustar los parámetros de TRACLUS para maximizar su efectividad en el contexto específico del proyecto.

Adicionalmente, se evaluó el uso de una implementación de TRACLUS disponible en una biblioteca externa, explorando su viabilidad y características. Aunque esta versión fue útil para los primeros experimentos, se consideraron también posibles variantes y adaptaciones del algoritmo, con el fin de entender mejor el alcance y la adaptabilidad de TRACLUS.

Todos estos estudios fueron necesarios para cumplir con el objetivo del proyecto: implementar el algoritmo TRACLUS y compararlo con variantes del mismo, como las que se detallan a continuación.

\subsubsection{Propuesta de variantes de TRACLUS}

Durante el desarrollo del proyecto, se analizaron varias propuestas de variantes del algoritmo TRACLUS que han surgido en estudios recientes. Aunque finalmente no se utilizaron, debido al excesivo tiempo que llevaría desarrollar las nuevas variantes, la investigación detallada de estos derivados fue enriquecedora y ayudó a contrastar enfoques para una implementación eficiente. Las variantes de TRACLUS estudiadas fueron:

\begin{itemize}
    \item \textbf{GTraclus:} Una variante diseñada para ejecutarse en unidades de procesamiento gráfico (GPU), optimizando la eficiencia del algoritmo al aprovechar la capacidad de procesamiento paralelo de las GPUs \cite{gtraclus}.

    \item \textbf{ST-TRACLUS:} Esta versión incorpora una dimensión temporal además de la espacial, permitiendo realizar agrupamientos espaciotemporales de trayectorias, lo cual mejora la calidad del análisis para datos donde el tiempo es una variable relevante \cite{st-traclus}.

    \item \textbf{ND-TRACLUS:} Una extensión de TRACLUS que permite realizar el clustering en espacios de n dimensiones, expandiendo su aplicabilidad a trayectorias de mayor dimensionalidad \cite{nd-traclus}.

    \item \textbf{Neighborhood-Based Trajectory Clustering:} Una alternativa basada en densidad local de vecindad en lugar de densidad global. Este enfoque busca mantener la eficiencia de TRACLUS mientras reduce la necesidad de múltiples parámetros de entrada \cite{nb-traclus}.

    \item \textbf{Adaptive Trajectory Clustering based on Grid and Density (ATCGD):} Un método que introduce una cuadrícula y criterios de densidad para el análisis de patrones móviles, buscando reducir la complejidad computacional y la carga de trabajo en la calibración de parámetros, especialmente en aplicaciones a gran escala como la de trayectorias de vehículos en sistemas de transporte inteligente \cite{atcgd}.
\end{itemize}

\subsubsection{Variación de las técnicas de clustering}

Paralelamente a la investigación de las variantes de TRACLUS, se estudiaron técnicas adicionales de clustering, dado que el algoritmo requiere la clusterización de segmentos para su evaluación final. Con el objetivo de evaluar la viabilidad y robustez de TRACLUS, se analizaron diversas técnicas de clustering, y tras descartar varias por inviabilidad, se optó finalmente por emplear los siguientes algoritmos ya incluidos en la librería de \texttt{scikit-learn} \cite{sklearn_cluster}

\begin{itemize}
    \item \textbf{DBSCAN:} Algoritmo basado en densidad que identifica clusters de alta densidad separados por regiones de menor densidad, adecuado para datos espaciales y resistente al ruido.
    
    \item \textbf{OPTICS:} Similar a DBSCAN, pero permite una sensibilidad ajustable a la densidad, lo que lo hace más flexible para analizar datos con densidades variadas.

    \item \textbf{HDBSCAN:} Variante jerárquica de DBSCAN que ajusta automáticamente los parámetros de densidad, simplificando su aplicación en datos de densidad variable.

    \item \textbf{Spectral Clustering:} Algoritmo basado en el análisis de valores propios que es particularmente útil en datos no lineales o con clusters de formas complejas.

    \item \textbf{Agglomerative Clustering:} Método jerárquico ascendente que agrupa iterativamente elementos basándose en la proximidad, resultando útil en análisis donde la estructura jerárquica es relevante.
\end{itemize}

Cada técnica fue investigada en términos de sus características, sensibilidad a la densidad, capacidad de manejar ruido y aplicabilidad a datos espaciales. La exploración de estas técnicas permitió seleccionar aquellas que mejor se adaptaran a las necesidades específicas del proyecto y que complementaran el uso de TRACLUS en el análisis de trayectorias.

\subsection{Página Web}

El diseño de la página web comenzó con un prototipo básico que definía las funciones principales a implementar. La estructura inicial incluía:  
\begin{itemize}
    \item Una página para visualizar los datos sin procesar en forma de mapas de trayectorias y mapas de calor.
    
    \begin{figure}[h!]
    		\centering
    		\includegraphics[width=0.5\textwidth]{img/prototipo_web1.png}
    		\caption{Código innecesario.}
	\end{figure}
	\FloatBarrier
    
    \item Una segunda página dedicada a la comparativa de \textit{clusters} mediante mapas interactivos.
    
    \begin{figure}[h!]
    		\centering
    		\includegraphics[width=0.5\textwidth]{img/prototipo_web2.png}
    		\caption{Código innecesario.}
	\end{figure}
	\FloatBarrier
    
    \item Una tercera página para analizar la relación entre trayectorias y \textit{clusters} generados por diferentes algoritmos.
    
    \begin{figure}[h!]
    		\centering
    		\includegraphics[width=0.5\textwidth]{img/prototipo_web3.png}
    		\caption{Código innecesario.}
	\end{figure}
	\FloatBarrier
\end{itemize}

El desarrollo requirió integrar la biblioteca Dash con herramientas adicionales como \texttt{CSS Grid} para gestionar el diseño de los componentes. Para optimizar tiempos de carga, se decidió prerenderizar imágenes y tablas, limitando las opciones disponibles a configuraciones predefinidas.

Durante el proyecto, el diseño evolucionó para mejorar la experiencia del usuario. La página inicial se dividió en dos: una para la carga de datos y otra para la visualización de trayectorias. Además, se añadieron nuevas funcionalidades, como un botón para descargar datos generados y pantallas adicionales que permitían seleccionar algoritmos de \textit{clustering} o reutilizar experimentos previos.

El menú desplegable inicial fue sustituido por una barra de navegación, lo que simplificó la interacción entre las diferentes páginas. También se eliminaron características que no aportaban valor significativo, como el zoom en los mapas, y se adoptó un enfoque basado en el modelo vista-controlador para estructurar el proyecto de manera más eficiente.

\section{Optimización}

El que podríamos considerar como el segundo punto más relevante, y sin duda uno de los aspectos críticos del proyecto, han sido los tiempos de ejecución. Debido a las grandes cantidades de datos a procesar, cada cálculo y visualización requerían una considerable cantidad de tiempo, por lo que la optimización de tiempos se convirtió en una prioridad clave para asegurar la viabilidad del proyecto.

\subsection{Visualización y dibujado de Mapas}

El primer desafió en el rendimiento fue la visualización de mapas, especialmente al intentar realizar representaciones detalladas y visualmente atractivas. Al analizar distintas bibliotecas de Python para dibujar y visualizar mapas, se probaron y compararon múltiples opciones combinadas entre si, como \texttt{pandas}, \texttt{geopandas}, \texttt{folium}, \texttt{folium.plugins}, \texttt{matplotlib.pyplot}, \texttt{matplotlib.colors}, \texttt{contextily}, \texttt{pyproj}, \texttt{seaborn}, \texttt{pydeck}, \texttt{shapely.geometry}, \texttt{sklearn.preprocessing} y \texttt{scipy.stats}.

Tras diversas pruebas de rendimiento y calidad visual, se optó por utilizar principalmente \texttt{contextily} y \texttt{matplotlib.pyplot} para la visualización de mapas, ya que estas bibliotecas ofrecieron la mejor relación entre rendimiento y calidad gráfica. Para la visualización de \textit{heatmaps}, se incluyeron \texttt{numpy} como las opciones más eficientes para generar mapas de calor. Sin embargo, para lograr un equilibrio óptimo entre rendimiento y visualización, se sacrificaron ciertas funcionalidades como el zoom en mapas interactivos y detalles visuales avanzados.


\subsection{Optimización del algoritmo}

Uno de los principales desafíos del proyecto fue el tiempo de procesamiento, debido a la gran cantidad de datos que debían manejarse. El tamaño y la complejidad de los datos afectaron tanto la carga inicial como la ejecución de algoritmos y la visualización, requiriendo una optimización constante para lograr resultados en tiempos razonables.

\subsubsection{Carga de Datos}

El primer obstáculo en términos de tiempo fue la carga de datos en la aplicación web, ya que \texttt{Dash} tiene limitaciones al cargar grandes volúmenes de datos. El archivo inicial pesaba aproximadamente dos gigabytes, y ni siquiera herramientas como Excel podían manejar correctamente un CSV de este tamaño, lo que llevó a errores frecuentes al intentar reducir su tamaño sin comprometer la integridad de los datos.

Tras varios intentos de conversión y reducción, se logró un tamaño adecuado que permitió cargar los datos en la aplicación. Además, para estructurar y analizar los datos en trayectorias geográficas, fue necesario procesarlos con \texttt{GeoDataFrame} mediante la biblioteca \texttt{GeoPandas}. Gracias a esta biblioteca, el tiempo de carga y procesamiento fue moderado y permitió manipular los datos de forma eficiente para futuras visualizaciones y análisis.

Para la visualización, el tamaño del conjunto de datos impactó en la calidad gráfica y los tiempos de carga. Fue necesario optar por bibliotecas de visualización que equilibraran la calidad y el tiempo de renderizado, sacrificando algunas características visuales avanzadas en favor de un rendimiento adecuado. Finalmente, se optó por \texttt{contextily} y \texttt{matplotlib.pyplot} para la representación de mapas y \texttt{numpy} para los mapas de calor.

\subsubsection{Distancias}

El proceso más complejo y exigente en términos de tiempo fue la ejecución del algoritmo TRACLUS. Durante su ejecución, es necesario calcular las distancias perpendicular, paralela y angular entre todas las trayectorias, lo que genera una complejidad algorítmica exponencial \(O(n^2)\). Esto se vuelve aún más costoso cuando se desean evaluar varias opciones de clustering en el mismo conjunto de datos.

Antes de las optimizaciones, el programa tardaba aproximadamente dos minutos y cuarenta y cinco segundos en cargar cien filas de datos, y una hora y cinco minutos en cargar quinientas filas, lo cual era insostenible dado que el archivo fuente de datos del proyecto contenía más de un millón de filas.

Para reducir estos tiempos, se realizaron múltiples pruebas y técnicas de optimización:

\begin{enumerate}
    \item \textbf{Numpy para la matriz de distancia:} Inicialmente, se intentó cargar la matriz de distancia utilizando \texttt{numpy}, que es una biblioteca altamente optimizada para cálculos matemáticos. Sin embargo, los resultados generados no coincidían con los obtenidos mediante el cálculo original, lo que condujo a diferencias significativas en los resultados de clustering.

    \item \textbf{Vectorización con Numpy:} En un segundo intento, se volvió a probar con \texttt{numpy}, aplicando un enfoque más completo de vectorización para las tres distancias. Nuevamente, aunque los tiempos de ejecución mejoraron, los resultados no fueron precisos, afectando la coherencia en los datos obtenidos.

    \item \textbf{Threading con bucles:} Para el tercer intento, se reutilizaron las funciones originales de cálculo de distancias, pero paralelizando los tres bucles de cálculo de distancias mediante \texttt{threading}. Esta prueba mejoró los tiempos levemente, reduciendo la carga de cien filas a dos minutos y veintidós segundos, manteniendo los resultados consistentes, aunque la mejora fue insuficiente para los objetivos del proyecto.

    \item \textbf{ThreadPoolExecutor y chunks:} En la cuarta prueba, se dividió la matriz de distancia en “chunks” para ser procesados en paralelo con \texttt{ThreadPoolExecutor}. Sin embargo, esta técnica solo generó una mejora marginal, con una reducción de aproximadamente seis segundos en la carga de cien filas, lo cual, aunque significativo para cantidades de datos pequeñas, resultó ineficiente para volúmenes mayores.

    \item \textbf{Paralelización manual y threading:} Finalmente, se probó una combinación de threading, sin usar \texttt{numpy} para los cambios en las funciones de distancia, calculando las distancias en tres hilos independientes. Este enfoque proporcionó el mejor resultado, logrando una reducción de tiempo a la mitad: un minuto y cuatro segundos para cien filas, y treinta y un minutos y cuarenta y nueve segundos para quinientas filas. Sin embargo, aunque esta reducción era significativa, los tiempos seguían siendo elevados para la totalidad de los datos del proyecto.
\end{enumerate}

Este proceso de optimización requirió numerosos ajustes y pruebas incrementales, en muchos casos con cambios mínimos y variaciones en el tamaño de los datos de prueba, lo que demandó una gran cantidad de horas de prueba y error sin obtener los resultados esperados. No se tiene un cálculo exacto, pero este apartado del proyecto consumió no solo decenas, sino cientos de horas dedicadas exclusivamente a la ejecución y análisis de pruebas.

En conclusión, aunque Python es un lenguaje versátil, sus limitaciones en paralelización y threading efectivo complicaron la optimización de cálculos intensivos en comparación con otros lenguajes como Java, lo que limitó el rendimiento alcanzable en este proyecto.


\subsection{Optimización de la página web} 

Tras múltiples ejecuciones de la página web para comparar los resultados de los diferentes algoritmos de clustering aplicados al TRACLUS, se identificó un problema recurrente: el tiempo de ejecución elevado.

En la aplicación, se podían ejecutar hasta cinco veces consecutivas el algoritmo TRACLUS, cada vez con modificaciones en el algoritmo de clustering. Esta ejecución lineal incrementaba significativamente los tiempos, ya que incluso con pocos datos, el algoritmo mostraba lentitud.

\subsubsection{Estrategia inicial: Uso de hilos}
Para abordar este problema, se reutilizó una estrategia previamente implementada: el uso de hilos. En el controlador \texttt{clustering.py}, que ya gestionaba correctamente el flujo de carga de datos, se dividió el código en funciones separadas, una por cada algoritmo de clustering. Estas funciones eran llamadas según las selecciones del usuario mediante la biblioteca \texttt{threading}.

\subsubsection{Problemas con la generación de mapas y tablas}
Posteriormente, se intentó incorporar la creación de mapas y tablas a estos hilos. Sin embargo, surgieron incompatibilidades con la biblioteca \texttt{matplotlib}, que no es compatible con librerías de hilos como \texttt{concurrent-} \texttt{.futures} o \texttt{threading}. Por este motivo, se decidió separar estas tareas. Una vez finalizados todos los hilos, se invocaba una nueva función encargada de generar los mapas necesarios para visualizar los datos.

\subsubsection{Limitaciones de Python}
Aunque esta optimización redujo el tiempo de ejecución, los resultados no alcanzaron las expectativas iniciales. Se identificó que la principal limitación era Python. Su configuración interna impide utilizar múltiples núcleos del procesador de manera eficiente, lo que restringía la velocidad máxima en dispositivos locales. Sin embargo, estas limitaciones no afectaban significativamente la aplicación en entornos remotos, donde el alto rendimiento no era una prioridad debido a los costes asociados.

\subsubsection{Exploración de alternativas de paralelización}
Dado el potencial escalamiento de la aplicación, se exploraron otras estrategias de paralelización:
- \texttt{asyncio}: Diseñada para tareas intensivas en I/O, pero no adecuada para este caso.
- \texttt{Batch Processing}: Orientada a flujos simples, tampoco era aplicable.
- \texttt{Dask} y \texttt{ProcessPoolExecutor}: Ambas opciones cumplían con los requisitos del proyecto.

Se optó por \texttt{ProcessPoolExecutor} debido a su menor complejidad y capacidad para ejecutar funciones en diferentes núcleos del procesador.

\subsubsection{Optimización del flujo de datos}
Además del uso de \texttt{multiprocessing}, se unificó el flujo de representación de los datos en las funciones de multiproceso, lo que resolvió incompatibilidades previas con \texttt{matplotlib}. Esto incrementó considerablemente el rendimiento, ya que anteriormente estas tareas se ejecutaban en serie.

\subsubsection{Resultados obtenidos}
Tras implementar estas mejoras, se lograron resultados significativos:
- \textbf{Operaciones singulares}: La ejecución de un solo TRACLUS con 100 filas de datos pasó de 312 segundos a 288 segundos.
- \textbf{Ejecuciones simultáneas}: La ejecución de cinco TRACLUS con sus respectivos algoritmos de clustering se redujo drásticamente, de 1800 segundos a 450 segundos.

Esta optimización no solo mejoró el rendimiento, sino que también eliminó errores previamente asociados al uso de hilos.


\section{Testing}

Para que un código sea verdaderamente funcional, seguro y mantenible, debe someterse a pruebas exhaustivas. Durante el desarrollo del proyecto, se implementaron varios tipos de pruebas para evaluar el rendimiento y la calidad del código, así como para ampliar el conocimiento en el campo del testing.

\subsection{Análisis de código estático}

El análisis de código estático es una técnica utilizada para identificar posibles errores en el código fuente sin necesidad de ejecutarlo. Este método es particularmente útil para evitar fallos humanos como bucles infinitos, errores de formato o problemas de nomenclatura.

Se evaluaron varias herramientas de análisis estático teniendo en cuenta las siguientes limitaciones:
\begin{itemize}
    \item Debían ser compatibles con Python y CSS, los lenguajes utilizados en el proyecto.
    \item No debían implicar costos adicionales.
    \item Era deseable la integración directa con GitHub para facilitar el flujo de trabajo.
\end{itemize}

Con estas características, se seleccionaron dos herramientas principales: \textbf{SonarQube} y \textbf{Code Climate}. De estas, \textbf{SonarQube} fue la más utilizada, ya que ofrecía soporte para analizar IPython Notebooks, un formato clave en los experimentos realizados durante el proyecto. Aunque los notebooks no formaban parte directa de la aplicación web, su análisis fue crucial para garantizar la calidad de los experimentos previos.

Lo ideal en este tipo de análisis es aplicarlo de manera continua durante las diversas fases del proyecto, ya que esto reduce significativamente la carga de trabajo y mejora la calidad del código a medida que se desarrolla. Sin embargo, en este proyecto, el análisis estático se realizó al final, lo que no fue óptimo pero permitió identificar una serie de problemas, entre ellos:

\begin{itemize}
    \item \textbf{Mejoras en nomenclatura:} Se ajustaron nombres de variables para facilitar su comprensión y evitar confusiones con otros datos.
    
\begin{figure}[h!]
    \centering
    \includegraphics[width=0.5\textwidth]{img/sonarq_regularexp.png}
    \caption{Error nomenclatura.}
    \label{fig:trayectorias_Spectral}
\end{figure}    
    
    \item \textbf{Variables no utilizadas:} Se detectaron y eliminaron variables que ya no eran relevantes, ya fuera por errores en su definición o por haber quedado obsoletas durante el desarrollo.
    
\begin{figure}[h!]
    \centering
    \includegraphics[width=0.5\textwidth]{img/sonarq_codigo_sobrante.png}
    \caption{Código innecesario.}
    \label{fig:trayectorias_Spectral}
\end{figure}

\begin{figure}[h!]
    \centering
    \includegraphics[width=0.5\textwidth]{img/sonarq_unused.png}
    \caption{Variable sin usar necesaria.}
    \label{fig:trayectorias_Spectral}
\end{figure}    
    
    \item \textbf{Importaciones innecesarias:} Se eliminaron módulos y librerías no utilizadas, lo que ayudó a reducir la carga del programa y mejorar su legibilidad.
    
\begin{figure}[h!]
    \centering
    \includegraphics[width=0.5\textwidth]{img/sonarq_imports.png}
    \caption{Importación excesiva.}
    \label{fig:trayectorias_Spectral}
\end{figure}
   
    \item \textbf{Complejidad en funciones:} Se identificaron funciones cuya complejidad excedía los límites recomendados. Aunque algunas de estas funciones no pudieron simplificarse debido a las características del algoritmo, el análisis ayudó a priorizar futuras mejoras.
    
\begin{figure}[h!]
    \centering
    \includegraphics[width=0.5\textwidth]{img/sonarq_complex.png}
    \caption{Complejidad grande.}
    \label{fig:trayectorias_Spectral}
\end{figure}

\end{itemize}

Aunque este tipo de análisis no corrige directamente errores en la funcionalidad del algoritmo, resulta muy útil para evitar problemas menores, mejorar la mantenibilidad del código y garantizar un nivel básico de seguridad.

\subsection{Pruebas unitarias y su implementación}

Las pruebas unitarias son fundamentales para garantizar la funcionalidad de cada componente del código y asegurar que los resultados generados sean consistentes y correctos. Durante el desarrollo de la aplicación, se planteó inicialmente realizar pruebas unitarias utilizando \texttt{pytest} que evaluaran el comportamiento de cada botón y función de la aplicación mientras esta se ejecutaba. El objetivo era cubrir todo el flujo de la web en tiempo real, desde la interacción del usuario con los botones hasta la ejecución de los \texttt{callbacks} de Dash.

Sin embargo, esta aproximación presentó múltiples problemas. Dash, como framework basado en componentes interactivos, no es directamente compatible con las herramientas de testing tradicionales. Intentar realizar pruebas unitarias mientras la aplicación se encontraba en ejecución resultó inviable, ya que las pruebas no se ejecutaban correctamente, y la interacción con los componentes de la interfaz gráfica no podía ser simulada adecuadamente. Como resultado, las pruebas terminaban siendo llamadas directas a las funciones asociadas a los botones, sin representar el comportamiento real del flujo de la aplicación ni garantizar que los \texttt{callbacks} funcionaran en contexto.

Ante esta limitación, se decidió cambiar el enfoque hacia pruebas más efectivas y relevantes. La nueva estrategia consistió en lo siguiente:

\begin{itemize}
    \item \textbf{Pruebas de funciones críticas:} Se probaron de manera individual las funciones más importantes de los modelos, como la implementación del algoritmo TRACLUS y su integración con diferentes algoritmos de clustering. Estas pruebas se realizaron utilizando conjuntos de datos generados de manera aleatoria para garantizar que el comportamiento del código fuese robusto ante diferentes escenarios.
    \item \textbf{Representación de mapas:} Se verificó que las funciones encargadas de generar mapas, tanto de segmentos como de clústeres, produjeran respuestas correctas. 
\end{itemize}










\capitulo{6}{Trabajos relacionados}

\section{Introducción}

En el desarrollo de este proyecto, fue crucial estudiar tanto los fundamentos teóricos como los trabajos previos relacionados con el algoritmo TRA-CLUS, ya que estos sirvieron como base para implementar y adaptar la solución propuesta. Este capítulo describe dos elementos clave: el artículo original que introduce TRA-CLUS y la biblioteca de código en GitHub que proporcionó una referencia práctica para la implementación del algoritmo.

\section{Estudio del algoritmo TRA-CLUS}

El algoritmo TRA-CLUS, introducido en el artículo \emph{Trajectory Clustering: A Partition-and-Group Framework} \cite{lee2007trajectory}, establece un marco novedoso para la agrupación de trayectorias basado en dos fases principales: segmentación y agrupamiento. Este enfoque busca identificar patrones significativos en conjuntos de datos espaciales y temporales, como los recorridos de taxis, rutas de navegación o trayectorias de animales.

Entre los conceptos fundamentales del algoritmo destacan:

\begin{itemize}
    \item \textbf{Segmentación}: Las trayectorias se dividen en segmentos lineales, utilizando un enfoque de optimización que minimiza el error de representación.
    \item \textbf{Agrupamiento}: Los segmentos resultantes se agrupan utilizando un algoritmo de clustering basado en densidad, como DBSCAN, para identificar patrones comunes en las trayectorias.
    \item \textbf{Representación}: Cada grupo de segmentos se representa mediante una trayectoria "representativa", que captura la esencia del grupo y permite una interpretación más clara de los datos.
\end{itemize}

Este marco de partición y agrupamiento demostró ser efectivo para manejar grandes volúmenes de datos geoespaciales, ofreciendo una solución escalable y precisa para el análisis de trayectorias. Este trabajo sirvió como base teórica para este proyecto, orientando el desarrollo y la implementación del algoritmo.

\section{Biblioteca TRA-CLUS}

Para complementar el estudio teórico del algoritmo, se utilizó como referencia práctica la biblioteca de código \emph{TRA-CLUS} disponible en GitHub \cite{traclus_library}. Esta implementación fue desarrollada por Adriel Amoguis, investigador asociado al Dr. Andrew L. Tan Data Science Institute (ALTDSI) \cite{altdsi}, una institución reconocida por su enfoque en proyectos avanzados de ciencia de datos y aprendizaje automático. La experiencia del autor en el desarrollo de herramientas de análisis de datos y su afiliación con una institución de prestigio hacen de esta biblioteca una fuente confiable y valiosa. Además, su código presenta una estructura modular y clara, lo que facilita su reutilización y adaptación para proyectos personalizados, cumpliendo con los estándares de calidad esperados en implementaciones de algoritmos complejos como TRA-CLUS.

\subsection{Características principales de la biblioteca}

La biblioteca \emph{TRA-CLUS} destaca por:

\begin{itemize}
    \item Una implementación directa de las dos fases principales del algoritmo: segmentación y agrupamiento.
    \item Código bien documentado que facilita su comprensión y modificación.
    \item Ejemplos prácticos que demuestran su aplicación en datasets sencillos, lo que reduce la curva de aprendizaje para los usuarios.
\end{itemize}

\subsection{Adaptaciones y mejoras realizadas en este proyecto}

Aunque la biblioteca original ofrecía una base sólida, fue necesario realizar una serie de adaptaciones para cumplir con los requisitos específicos de este proyecto:

\begin{itemize}
    \item \textbf{Optimización}: Se mejoraron ciertos aspectos del código para garantizar su rendimiento en conjuntos de datos más grandes, como los datos de trayectorias de taxis utilizados en este trabajo.
    \item \textbf{Integración con visualización}: Se desarrollaron herramientas adicionales para vincular los resultados del algoritmo con representaciones gráficas interactivas, facilitando la interpretación de los datos.
    \item \textbf{Extensibilidad}: Se modularizó aún más el código para permitir su integración en la aplicación web desarrollada.
    \item \textbf{Comparativas}: Se añadieron funciones para realizar comparaciones con otros algoritmos de clustering, como DBSCAN, OPTICS y HDBSCAN.
\end{itemize}

\section{Impacto en el proyecto}

El uso combinado del artículo original de TRA-CLUS y la biblioteca de código permitió acelerar el desarrollo y centrar los esfuerzos en aspectos diferenciadores del proyecto, como la visualización y el análisis comparativo. Además, este enfoque demostró cómo los trabajos previos pueden ser un recurso invaluable en la investigación y el desarrollo, ofreciendo tanto una base teórica sólida como herramientas prácticas para la implementación.

\capitulo{7}{Conclusiones y Líneas de trabajo futuras}

\section{Conclusiones}

El desarrollo de este proyecto ha permitido abordar diversas problemáticas relacionadas con el análisis y la visualización de datos geoespaciales mediante técnicas de clustering. A continuación, se resumen las principales conclusiones obtenidas durante su realización:

\begin{itemize}
    \item Se logró implementar con éxito el algoritmo TRA-CLUS, permitiendo segmentar y agrupar trayectorias geográficas de manera eficiente, y generando resultados útiles para la interpretación de grandes volúmenes de datos.
    \item La aplicación web desarrollada ha demostrado ser una herramienta valiosa para visualizar, comparar y analizar los resultados de diferentes algoritmos de clustering. La posibilidad de interactuar con gráficos, mapas y tablas ha facilitado la interpretación de los datos.
    \item Durante las pruebas comparativas, se evidenció que cada algoritmo de clustering tiene ventajas y limitaciones dependiendo de las características de los datos y los parámetros configurados. Esto refuerza la importancia de disponer de herramientas flexibles que permitan explorar múltiples opciones.
    \item La adopción de la arquitectura modelo-vista-controlador (MVC) mejoró significativamente la organización y mantenibilidad del código, facilitando la incorporación de nuevas funcionalidades y la resolución de problemas técnicos.
    \item El despliegue de la aplicación en un entorno de producción a través de la plataforma Render permitió validar la funcionalidad del sistema en condiciones reales, garantizando su accesibilidad a otros usuarios.
\end{itemize}

Técnicamente, el proyecto ha permitido afianzar conocimientos en áreas como el análisis de trayectorias, la programación en Python, el uso de bibliotecas de visualización y frameworks web, además de explorar técnicas de optimización y manejo de datos masivos.

\section{Líneas de Trabajo Futuras}

A pesar de los logros alcanzados, existen múltiples aspectos que podrían mejorarse o extenderse en futuros trabajos relacionados con este proyecto. Algunas de las líneas de trabajo futuras incluyen:

\begin{itemize}
    \item \textbf{Optimización del rendimiento}: Implementar técnicas de optimización tanto en el procesamiento de datos como en la generación de gráficos interactivos para reducir los tiempos de carga y procesamiento en datasets más grandes.
    \item \textbf{Ampliación de funcionalidades}: Incluir soporte para nuevos algoritmos de clustering, como Birch o K-Means, y permitir configuraciones más avanzadas de los parámetros por parte del usuario.
    \item \textbf{Análisis en tiempo real}: Adaptar la aplicación para procesar datos en tiempo real, lo que sería especialmente útil en aplicaciones relacionadas con la gestión del tráfico o la monitorización de flotas de vehículos.
    \item \textbf{Mejoras en la interfaz de usuario}: Refinar la experiencia del usuario mediante el uso de bibliotecas avanzadas de diseño y optimización de la navegación, además de añadir herramientas de análisis más intuitivas.
    \item \textbf{Integración con otras fuentes de datos}: Permitir la carga y análisis de datos provenientes de diversas fuentes, como APIs de mapas en tiempo real, sensores o bases de datos externas.
    \item \textbf{Validación con expertos}: Realizar evaluaciones cualitativas con expertos en análisis geoespacial para identificar posibles mejoras en los resultados generados y la interfaz de la aplicación.
    \item \textbf{Publicación científica}: Documentar los resultados obtenidos y presentarlos en conferencias o revistas científicas relacionadas con el análisis de datos y Big Data.
\end{itemize}

En conclusión, este proyecto ha sentado las bases para el desarrollo de sistemas de análisis de trayectorias avanzados, demostrando la viabilidad y utilidad de combinar algoritmos de clustering con herramientas de visualización. Los resultados obtenidos abren un abanico de posibilidades para futuras investigaciones y desarrollos en el ámbito del Big Data aplicado a datos geoespaciales.



\bibliographystyle{plain}
\bibliography{bibliografia}

\end{document}
